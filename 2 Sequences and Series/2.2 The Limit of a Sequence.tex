\subsection{The Limit of a Sequence}

\ex{1}
\begin{enumerate}[label=(\alph*)]
    \item 
    $\lim \frac{1}{6n^2+1} = 0$

    Scratchwork: We require that for some $\varepsilon > 0$
    \begin{align*}
        \frac{1}{6n^2+1} &< \varepsilon \\
        \Rightarrow n &> \sqrt{\frac{1}{6}\left( \frac{1}{\varepsilon} - 1 \right)}
    \end{align*}
    So we choose $N > \sqrt{\frac{1}{6}\left( \frac{1}{\varepsilon} - 1 \right)}$.

    \begin{proof}
        Let $\varepsilon>0$ be arbitrary. 
        For $n \geq N > \sqrt{\frac{1}{6}\left( \frac{1}{\varepsilon} - 1 \right)}$ where $n, N\in \mathbb N$
        then
        \begin{align*}
            \left| \frac{1}{6n^2+1}\right| &< \left| \frac{1}{6N^2+1}\right| \\
                                            &< \varepsilon
        \end{align*}
        
    \end{proof}

    \item 
    $\lim \frac{3n+1}{2n+5} = 3/2$

    \begin{proof}
        Let $\varepsilon>0$ be arbitrary. 
        For $n \geq N > \frac{1}{4}\left( \frac{13}{\varepsilon} - 10 \right)$ where $n, N\in \mathbb N$
        then
        \begin{align*}
            \left|  \frac{3n+1}{2n+5} - \frac{3}{2} \right| < \varepsilon
        \end{align*}
    \end{proof}

    \item
    $\lim \frac{2}{\sqrt{n+3}} = 0$

    \begin{proof}
        Let $\varepsilon>0$ be arbitrary. 
        For $n \geq N > \frac{4}{\varepsilon^2}-3$ where $n, N\in \mathbb N$
        then
        \begin{align*}
            \left|  \frac{2}{\sqrt{n+3}} \right| < \varepsilon
        \end{align*}
    \end{proof}
\end{enumerate}

\ex{2}
Consider this sequence which is vercongent but divergent
\begin{align*}
    (-1,1,-1,1,...)
\end{align*}
The defintion of vercongence is simply that the sequence be bounded above and 
below. The bound is called $\varepsilon$.

\ex{3}
\begin{enumerate}[label=(\alph*)]
    \item Find a college in the US with no students above seven feet.
    \item Find a college in the US where every professor has given a
    grade other than an A or B.
    \item Find a student at every US college which is below six feet.
\end{enumerate}

\ex{4}
The greatest distance between the largest and smallest terms of the sequence
 are $0$ and $1$ so for any $\varepsilon > 1$ then 
 \begin{align*}
    |a_n - 0| \leq 1 < \varepsilon
 \end{align*}
 for any $n \leq N = 1$ where $n, N \in \mathbb{N}$. Conversely, for $\varepsilon \leq 1$
 there exists no suitable response.

 \ex{5}
 \begin{enumerate}[label=(\alph*)]
    \item 
    Observe $a_n = 0$ for $n > 1$ so 
    \begin{align*}
        |a_n - 0| = 0 \quad \forall n \geq N > 1 
    \end{align*}
    So $(a_n) \rightarrow 0$.

    \item 
    Observe, for $n>10$ $10+n < 2n$ so the tail of the sequence is imply zeros so
    $N = 11$ is a suitable response to any $\varepsilon > 0$.
 \end{enumerate}

\ex{6}
\begin{enumerate}[label=(\alph*)]
    \item larger
    \item larger
\end{enumerate}

\ex{7}
\begin{enumerate}[label=(\alph*)]
    \item 
    \begin{definition}
        A sequence $(a_n)$ converges to infinity if for every 
        $\varepsilon > 0$, there exists $N \in \mathbb{N}$ s.t. 
        for all $n \geq N$ it follows $|a_n| > \varepsilon$.
    \end{definition}

    We now prove $(\sqrt n)$ converges to infinity.
    \begin{proof}
        Let $\varepsilon > 0$ be arbitrary, then
        \begin{align*}
            |\sqrt n | > \varepsilon \quad \forall n \geq \mathbb{N} > \varepsilon
        \end{align*}
    \end{proof}

    \item 
    This sequence does not converge to infinity since oscillates.
\end{enumerate}

\ex{8}
\begin{enumerate}[label=(\alph*)]
    \item frequently
    \item Eventually is stronger than frequently since eventually 
    $\Rightarrow $ frequently.

    \item
    \begin{definition}
        A sequence $(a_n)$ converges to a if, given any $\varepsilon$-neighbourhood
         $V_\varepsilon(a)$, $(a_n)$ is eventually in $V_\varepsilon(a)$.
    \end{definition}

    \item 
    It is frequently but not necessarily eventually in $(1.9, 2.1)$.
    For instance, consider the sequence
    \begin{equation*}
        (x_n) = (2,0,2,0,2,0,...)
    \end{equation*}
\end{enumerate}


