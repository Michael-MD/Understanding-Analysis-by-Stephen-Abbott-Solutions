\subsection{The Algebraic and Order Limit Theorems}

\ex{1}
\begin{proof}
    Let $\varepsilon > 0$ be arbitrary. Then 
    \begin{align*}
        |a_n - a| = 0 < \varepsilon \quad \forall n \geq \mathbb{N} = 1
    \end{align*}
\end{proof}

\ex{2}
\begin{enumerate}[label=(\alph*)]
    \item 
    (a) is a special case of (b) so we only prove (b).

    \item 
    \begin{align*}
        x &= \lim x_n \\
        &= \lim{\sqrt{x_n} \sqrt{x_n}} \\
        &= \lim \sqrt{x_n} \lim \sqrt{x_n} \\
        &= (\lim \sqrt{x_n})^2 
    \end{align*}
    Hence, $\lim \sqrt{x_n} = \sqrt x$.
\end{enumerate}


\ex{3}
By the order limit theorem,
\begin{align*}
    x_n \leq y_n \leq z_n \Rightarrow \lim x_n \leq \lim  y_n \leq \lim z_n
\end{align*}
So if $\lim x_n = \lim z_n = l$ then $\lim y_n = l$.

\ex{4}
\begin{align*}
    0 &= \lim 0 = \lim{a_n - a_n} \\
    &= \lim a_n - \lim a_n \\
    &= l_1 - l_2 \\
    \Rightarrow l_1 &= l_2
\end{align*}


\ex{5}
\begin{proof}
    $(\Rightarrow)$ For arbitrary $\varepsilon > 0$ choose $N \in \mathbb{N}$
    s.t. 
    \begin{align*}
        |z_n - z| < \varepsilon \quad \forall n \geq N
    \end{align*}

    Define 
    \begin{align*}
        N_1 = 
        \begin{cases}
            (N+2)/2 & N \text{ even} \\
            (N+1)/2 & N \text{ odd} \\
        \end{cases}
    \end{align*}
    so for $n_1 \geq N_1$
    \begin{align*}
        |x_{n_1} - z| = |z_{2n_1-1} - z| < \varepsilon
    \end{align*}
    and
    \begin{align*}
        |y_{n_1} - z| = |z_{2y_1} - z| < \varepsilon
    \end{align*}
    so $\lim x_n = z = \lim y_n$.

    $(\Leftarrow)$ Suppose $(x_n) \rightarrow z, (y_n) \rightarrow z$, then
    \begin{align*}
        |z_n - z| = 
        \begin{cases}
            |x_n - z| & n \text{ odd} \\
            |y_n - z| & n \text{ even}
        \end{cases}
    \end{align*}
    Now $\exists$ $N_1, N_2 \in \mathbb{N}$ s.t. $\forall$ $n_1 \geq N_1$
    and $n_2 \geq N_2$
    \begin{align*}
        |x_n-z| &< \varepsilon \\
        |y_n-z| &< \varepsilon
    \end{align*}
    so choose $N = \max \{N_1, N_2\}$ so that 
    \begin{align*}
        |z_n-z| < \varepsilon
    \end{align*}
    so $\lim z_n = z$.
\end{proof}

\ex{6}
\begin{enumerate}[label=(\alph*)]
    \item 
    \begin{proof}
        Let $\varepsilon > 0$. Next choose $N=\max\{N_1, N_2\}$ s.t.
        \begin{align*}
            |b_{n_1}-b| < \varepsilon \quad \forall n_1 \geq N_1
        \end{align*}
        and  
        \begin{align*}
            |b_{n_2}-b| < |b| \quad \forall n_2 \geq N_2
        \end{align*}
        So we are guaranteed to be far enough into the sequence such that
        the remaining terms match the sign of $b$ i.e. $\text{sgn } b_n = \text{sgn } b$.
        So
        \begin{align*}
            \varepsilon > |b_n-b| = ||b_n| - |b||
        \end{align*}
        So $(|b_n|) \rightarrow b$.
    \end{proof}

    \item 
    The converse of (a) is not true. 
    Take the sequence $(b_n) = \{1,-1,1,-1,...\}$ which is divergent but
    $(|b_n|) \rightarrow 1$.
\end{enumerate}

\ex{7}
\begin{enumerate}[label=(\alph*)]
    \item 
    \begin{proof}
        Let $\varepsilon > 0$ be arbitrary. Then choose
         $N \in \mathbb{N}$ s.t. 
         \begin{align*}
            |b_n| < \varepsilon / M \quad \forall n\geq N
         \end{align*}
         where $M = \sup a_n$. Then
         \begin{align*}
            |a_nb_n| \leq M |b_n| < M \frac{\varepsilon}{M} < \varepsilon
         \end{align*}
    \end{proof}
    
    We cannot use the algebraic limit theorm since both Sequences
    need to converge but we cannot assume this about $(a_n)$.

    \item 
    If $(b_n)$ converges to a non-zero limit then no general conclusion can 
    be made since if $(a_n) =\{ 1,-1,1,-1,...\}$ then the sequence $(a_nb_n)$
    would oscillate between $b$ and $-b$ but never converge i.e. there is no
    response for example for $\varepsilon = |b|$.

    \item
    \begin{proof}
        Since we assume $(a_n) \rightarrow 0$ and
        $(1/b_n)$ is bounded then by (a) $(a_nb_n) \rightarrow 0$.
    \end{proof}
\end{enumerate}

\ex{8}
\begin{enumerate}[label=(\alph*)]
    \item 
    Sequences $(x_n), (y_n)$ where $x_n=n, y_n=-n$ diverge but 
    $(x_n+y_n) = (0)$ converges.

    \item 
    Impossible, since $y_n = (x_n + y_n) - x_n$ so by the algebraic limit 
    theorem
    \begin{align*}
        \lim y_n = \lim{x_n+y_n} - \lim x_n
    \end{align*}
    since the RHS converges then the LHS also converges.
    
    \item
    Consider sequence $(b_n) $ where $b_n = 1/n$.

    \item 
    Impossible since $(b_n)$ converges then it is bounded also. Since $(a_n-b_n)$ is bounded
    then this would imply $(a_n)$ is bounded also.

    \item 
    This is simply \Ex{7}. Consider $(b_n) = \{1,-1,1,-1,...\}$ and 
    $(a_n)\rightarrow 0$ then $(a_nb_n)$ will surely converge.
\end{enumerate}

\ex{9}
If we try to assume stricit inequalities then none of the statements of the
order limit theorem continue to hold. We provide counter examples for each.
\begin{enumerate}[label=(\roman*)]
    \item 
    Consider $(a_n)\rightarrow a$ where $a_n = 1/n > 0$ but $a=0$.
    
    \item 
    This case reduces to the previous if we assume $a_n=0$ for all $n$.

    \item 
    If $c=0$ and $c = 0 < b_n$ for all $n$. This is reduces to the first case.
    Suppose $a_n <c =  0$ for all $n$. Then $-a_n > 0$ which reduces to the 
    first case again.
\end{enumerate}

\ex{10}
\begin{proof}
    For $\varepsilon > 0$. Choose $N \in \mathbb{N}$ s.t. 
    \begin{align*}
        |a_n| < \varepsilon \quad \forall n \geq N
    \end{align*}
    Then
    \begin{align*}
        |b_n - b| < |a_n| < \varepsilon
    \end{align*}
\end{proof}

\ex{11}
\begin{proof}
    Let $\varepsilon > 0$ be arbitrary.
    Observe $x_n < M \, \forall n$ then by the order limit theorem $|x_n - x| \, \forall n$.
    So we choose $N > M \ varepsilon$ then
    \begin{align*}
        |y_n - x| \leq \sum_m^n \frac{|x_n-x|}{n} < \frac{M}{M/\varepsilon} < \varepsilon
    \end{align*}

    The second equality follows since
    \begin{align*}
        |y_n - x| &=  \left| \frac{1}{n}\sum_m^n x_m - x\right| \\
        &= \frac{1}{n} \left| \sum_m^n x_m - nx \right| \\
        &= \frac{1}{n} \left| \sum_m^n (x_m - x) \right| \\
        &\leq \frac{1}{n}  \sum_m^n \left|x_m - x \right|
    \end{align*}
\end{proof}

\ex{12}
\begin{enumerate}[label=(\alph*)]
    \item 
    We expect $\lim_{m,n}a_{m,n}$ to represent what happens to the tail 
    of hte doubly indexed sequence. 

    \begin{align*}
        \lim_n \lim_m a_{m,n} &= 1 \\
        \lim_m \lim_n a_{m,n} &= 0
    \end{align*}

    \item 
    \begin{definition}
        $(a_{m,n}) \rightarrow a$ if $\forall \, \varepsilon > 0 \, \exists$
        $N \in \mathbb{N}$ s.t. 
        \begin{align*}
            |a_{m,n} - a| < \varepsilon \quad \forall m,n \geq N
        \end{align*} 
    \end{definition}
\end{enumerate}
