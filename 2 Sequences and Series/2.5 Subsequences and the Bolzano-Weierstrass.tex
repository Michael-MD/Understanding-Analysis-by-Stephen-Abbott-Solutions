\subsection{Subsequences and the Bolzano-Weierstrass Theorem}

\ex{1}
\begin{proof}
    Suppose $(x_n)$ is convergent. Consider subsequence $(x_{n_j})$,
    then for $\varepsilon > 0$ $\exists$ $N\in \mathbb{N} $ s.t. 
    \begin{align*}
        |x_n-x| < \varepsilon
    \end{align*}
    for $n \geq N$. However, this also implies
    \begin{align*}
        |x_{n_j} - x| = |x_{n} - x| < \varepsilon
    \end{align*}
    for $n_j \geq N$.
\end{proof}

\ex{2}
\begin{enumerate}[label=(\alph*)]
    \item 
    \begin{proof}
        Since $\sum a_n$ converges, then any subsequence of partial sums of
        $\sum a_n$ converge. In particular, observe that any regrouping of 
        terms 
        \begin{align*}
            &(a_1+a_2+ \cdots + a_{n_1}) \\ &+ (a_{n_1+1}+a_{n_1+2}+ \cdots + a_{n_2}) \cdots
        \end{align*}
        corresponds to subsequence of partial sums $\{s_{n_1},s_{n_2},...\}$.
        which converges to the same limit.
    \end{proof}

    \item
    Since the partial sequences diverge then any convergent partial sequence
    is not guaranteed to converge to the same value. We happened to 
    find a partial sequence in section 2.1 which converges.
\end{enumerate}

\ex{3}
\begin{enumerate}[label=(\alph*)]
    \item 
    Consdier sequence
    \begin{align*}
        \{1-\frac{1}{2}, \frac{1}{2}, 1-\frac{1}{3}, \frac{1}{3}, ...\}
    \end{align*}
    where $(a_{2n+1})$ converges to 0 and $(a_{2n})$ converges to 
    1.

    \item 
    Any monotone sequence needs to be unbounded to diverge so
    the partial sums need to be unbounded also.

    \item 
    Consider
    \begin{align*}
        \left\{1, \frac{1}{2}, 1, \frac{1}{2}, \frac{1}{3}, 1, \frac{1}{2}, \frac{1}{3}, \frac{1}{4},..\right\}
    \end{align*}

    \item
    Consider sequence
    \begin{align*}
        \{0,1,0,2,0,3,0,4,...\}
    \end{align*}

    \item
    Impossible since by the Bolzano-Weierstrass \Thm a bounded 
    subsequence will 
    contain a convergent subsequence of its own. This sequence is also embedded
    in the original sequence.
\end{enumerate}

\ex{4}
Suppose $(a_n)$ does not converge to $a$. So there exists 
 some $\varepsilon$ s.t. for every $N\in \mathbb{N}$ there exists $n \geq N$ where
\begin{align*}
    |a_n - a| > \varepsilon
\end{align*}
We can construct subsequence $(a_{n_k})$ as follows
\begin{itemize}
    \item Choose $N = 1$ and set $a_{n_1}$ s.t. $|a_{n_1} - a| > \varepsilon$ 
    \item Choose $N = n_1 + 1$ and repeat this process
\end{itemize} 
Hence, every term of $a_{n_k}$ will satisfy $|a_{n_k} - a| > \varepsilon$.
However, this seqeunce is bounded and by the Bolzano-Weierstrass \Thm there 
exists a convergent subsequence. By assumption this subsequence 
 converges to $a$. However this would imply there exists $N$ s.t. 
 $|a_{n_k} - a| < \varepsilon$ for $n_k\geq N$ which violates our starting
 assumption. Hence, $(a_n)\rightarrow a$.


\ex{5}
$b=0$ trivially converges so lets consider $-1<c<0$ where $c=-b$.

Since we know $(|c_n|)$ converges then $\exists N$ s.t. 
\begin{align*}
    \varepsilon > ||c_n|| = |(-1)^n b^n| = |(-b)^n|
\end{align*}

\ex{6}
Now since $s=\sup S$, then 
\begin{align*}
    s - \varepsilon < x \in S
\end{align*}
so there are infinitely $a_n$ greater than $s - \varepsilon$.
Since $s$ is an upper bound then $s+\varepsilon$ is also an upper bound. Hence,
there exist a finite number of elemts of $(a_n)$ greater than $s+\varepsilon$
so there exist an infinite number of elements of $(a_n)$ s.t. $a_n < \varepsilon$.
So there exists an infinite number of elments in the interval 
\begin{align}
    (s-\varepsilon, s+\varepsilon)
\end{align}
We will use this to find a convergent subsequence. 
Choose $a_{n_k}$ in the interval $(s - 1/k,s+1/k)$ s.t. $n_{k+1}>n_k$.
Then it is clear that $(a_{n_k}) \rightarrow s$.