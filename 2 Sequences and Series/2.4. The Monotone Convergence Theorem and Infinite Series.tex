\subsection{The Monotone Convergence Theorem and Infinite Series}

\ex{1}
\begin{theorem}
    Suppose $(bn)$ is decreasing and satisfies $bn \geq 0$ for all $n \in N$. 
    If $(b_n)$ converges then
    \begin{align*}
        \sum_n^\infty 2^n b_n = b_1 + 2b_2 + 4b_4 + 8b_8 + 16b_{16} + \cdots
    \end{align*}
    converges
\end{theorem}
\begin{proof}
    Let $s_n, t_n$ be the partial sums of $\sum_m b_m$ and $\sum_m 2^m b_m$ respectively. 
    We will prove the contrapositive of this statement. So we aim to prove that if 
    $(t_n)$ diverges then so does $(s_n)$.
    Observe
    \begin{align*}
        s_{2^{k+1}} &= b_1 + (b_2) + (b_3 + b_4) + (b_5 + b_6 + b_7 + b_8) \\
        & + \cdots + (b_{2^k+1} + \cdots + b_{2^{k+1}}) \\ 
        &\geq b_1 + b_2 + 2b_4 + 4b_8 + \cdots + 2^{k}b_{2^{k+1}}\\
        &= \frac{b_1}{2} + \frac{1}{2}\left( b_1 + 2b_2 + \cdots + 2^{k+1} b_{2^{k+1}} \right) \\
        &= \frac{b_1}{2} + \frac{t_{2^{k+1}}}{2}
    \end{align*}
    So if $(t_n)$ diverges and we know the sequence is 
    monotone increasing then it must be unbounded, this implies $(s_n)$ is also unbounded
    and therefore cannot converge.
\end{proof}

\ex{2}
\begin{enumerate}[label=(\alph*)]
    \item 
    \begin{proof}
        We prove this statement by proving the sequence is monotone and bounded.
        
        We begin by proving monotonicity using induction.
        \begin{itemize}
            \item Base case: For $n=1$, $x_2 = 3 \leq x_1 = 1$.
            \item Inductive step: For $n>1$, by the induction hypothesis $x_n \leq x_{n-1}$.
            This implies 
            \begin{align*}
                x_n = 4 - \frac{1}{n_{n+1}} &\leq x_{n-1} = 4 - \frac{1}{n_{n}} \\
                \Rightarrow \frac{1}{x_{n+1}} &\geq \frac{1}{x_n} \\
                \Rightarrow x_{n+1} &\leq x_n
            \end{align*}
        \end{itemize}
    \end{proof}

    \item 
    $(x_{n+1})$ and $(x_n)$ have the same tail and only the tail behaviour 
    influences the limit.

    \item 
    \begin{align*}
        x = \lim x_{n+1} &= \lim \frac{1}{4-x_n} \\
                    &= \frac{1}{4-\lim x_n} \\
                    &=         \frac{1}{4-x} \\
                    \Rightarrow x = 2\pm \sqrt 3
    \end{align*}
    Since $x_n \leq 3$ $\lim x_n = 2 - \sqrt 3$.
\end{enumerate}

\ex{3}
\begin{proof}
    We will prove this result using the monotone convergenec \Thm.
    First we prove the sequence is monotonic increasing using 
    induction.

    \begin{itemize}
        \item Base case: $y_2 \geq y_1 $ $\Rightarrow$ $\frac{1}{y_2} \leq \frac{1}{y_1}$
        $\Rightarrow$ $y_3 = 4 - \frac{1}{y_2} \geq 4 - \frac{1}{y_1} = y_2$.
        So $y_3 \geq y_2$.

        \item Inductive step: $y_n \geq y_{n-1} $ $\Rightarrow$ $\frac{1}{y_{n}} \leq \frac{1}{y_{n-1}}$
        $\Rightarrow$ $y_{n+1} = 4 - \frac{1}{y_n} \geq 4 - \frac{1}{y_{n-1}} = y_n$.
        So $y_{n+1} \geq y_n$.
    \end{itemize} 
    The result follows by the induction hypothesis.

    Since the sequence, $(y_n)$, is bounded below by $0$ and so $\frac{1}{y_n} \geq 0$
     so $y_n = 4-\frac{1}{y-{n-1}}\leq 4$. 
     Then by the Monotone convergence \Thm the sequence is convergent.
\end{proof}

We can now proceed to find the limit 
\begin{align*}
    y = \lim y_{n+1} &= \lim 1 - \frac{1}{y_n} \\ 
                &= 1 - \frac{1}{\lim y_n} \\
                &= 1 - \frac{1}{y} \\
                & \Rightarrow y = 2\pm \sqrt 3
\end{align*}
By the order limit \Thm since $y_n \geq 1 \forall n \in \mathbb{N}$ then $y \geq 1$ so $y = 2+\sqrt 3$.  

\ex{4}
\begin{proof}
    Let $(z_n)$ be the sequence of interest. First we prove $(z_n)$ is bounded above by $2$ using
    induction.
    \begin{itemize}
        \item Base step: $z_1 \leq 2$
        \item Inductive step: $z_{n+1} \leq \sqrt 2 \sqrt{z_{n}}$ but $\sqrt{z_{n}} \leq \sqrt 2$ 
        $\Rightarrow$ $z_{n+1}\leq \sqrt 2 \sqrt{\sqrt 2} \leq \sqrt 2 \sqrt 2 = 2$.
    \end{itemize}
    It follows by the induction hypothesis that $z_n \leq 2 \forall n \in \mathbb{N}$ 
    so $(z_n)$ is bounded.

    Next we show the sequence is monotonic increasing, this is easy to see if we obtain an explicit formula
    for $(z_n)$ given by 
    \begin{align*}
        z_n = \prod_{m=1}^n 2^{\frac{1}{2^m}}
    \end{align*}
    So 
    \begin{align*}
        z_{n+1} &= \prod_{m=1}^{n+1} 2^{\frac{1}{2^m}} \\
                &= 2^{\frac{1}{2^{n+1}}}\prod_{m=1}^{n} 2^{\frac{1}{2^m}} \\
                &= 2^{\frac{1}{2^{n+1}}} z_n
    \end{align*}
    Since $2^{\frac{1}{2^{n+1}}} \geq 1 \Rightarrow z_{n+1} \leq z_n$. Hence by the Monotone convergence
    theorem $(z_n)$ converges.
\end{proof}

\ex{5}
\begin{enumerate}[label=(\alph*)]
    \item 
    \begin{proof}
        First we prove $x_n^2 > 2$. Now $n=1$, $x_1^2=2^2 = 4 > 2$.
        For $n > 1$
        \begin{align*}
            x_n^2 - 2 &= \frac{1}{2}\left( x_{n-1} + \frac{2}{x_{n-1}} \right)^2 - 2 \\
                    &= \frac{1}{4} \left( x_{n-1}^2 + \frac{4}{x_{n-1}^2} - 4 \right) \\
                    &= \frac{1}{4}\left( x_{n-1} - \frac{2}{x_{n-1}} \right)^2 \\
                    &> 0
        \end{align*}
        $x_{n} > 2$.

        Next we prove $x_n - x_{n+1} \geq 0$:
        \begin{align*}
            x_n - x_{n+1} &= x_n - \frac{1}{2}\left( x_n + \frac{2}{x_n} \right) \\
                        &= \frac{-1}{2}\left( -x_n + \frac{2}{x_n} \right) \\
                        &= \frac{1}{2x_n}\left( x_n^2 - 2 \right) \\
                        &\geq 0
        \end{align*}
        Since $x_n > 0$ and $x_n^2 > 2 \, \forall n \in \mathbb{N}$.
        Since the sequence is monotone decreasing and bounded it must converge.
        Taking the limit on both sides we obtain
        \begin{align*}
            x = \frac{1}{2}\left( x + \frac{2}{x} \right)
        \end{align*}
        so $x = \sqrt 2$.
    \end{proof}

    \item 
    If we modify the sequence s.t. 
    \begin{align*}
        x_{n+1} = \frac{1}{2}\left( x_n + \frac{c}{x+n}\right)
    \end{align*}
    where $x_1=c$.
    and we follow the steps as we did in (a) we will find $(x_n) \rightarrow \sqrt c$.
\end{enumerate}

\ex{6}
\begin{enumerate}[label=(\alph*)]
    \item 
    \begin{proof}
        Let $T_n = \{ a_k : k \geq n \}$.

        Now $T_{m+1}$ is the same as $T_m$ with $a_m$ removed so 
        \begin{itemize}
            \item If $a_m \leq a_n $ for some $a_n \in T_{n+1}$ then 
            $\sup T_n = \sup T_{n+1}$.
            
            \item If $a_m \geq a_n $ for all $a_n \in T_{n+1}$ then 
            $\sup T_n \geq \sup T_{n+1}$.
        \end{itemize}
        Hence $y_{n+1} \leq y_n$ so $(y_n)$ is monotonic.

        Now $(a_n)$ being bounded implies $(T_n)$ is bounded since
        $T_n \subseteq (a_n)$. Hence, 
        \begin{align*}
            |y_n| = |\sup T_n| < M
        \end{align*}
        where $M$ is an upper bound for $(x_n)$ then $(y_n)$ is bounded.

        Since $(y_n)$ is monotonic and bounde above, by the
         Monotone convergence \Thm
         $(y_n)$ converges.
    \end{proof}

    \item 
    \begin{proof}
        Let $T_n = \{ a_k : k \geq n \}$.

        Now $T_{m+1}$ is the same as $T_m$ with $a_m$ removed so 
        \begin{itemize}
            \item If $a_m \geq a_n $ for some $a_n \in T_{n+1}$ then 
            $\inf T_n = \inf T_{n+1}$.
            
            \item If $a_m \leq a_n $ for all $a_n \in T_{n+1}$ then 
            $\sup T_n \leq \sup T_{n+1}$.
        \end{itemize}
        Hence $x_{n+1} \geq x_n$ so $(y_n)$ is monotonic.

        Now $(a_n)$ being bounded implies $(T_n)$ is bounded since
        $T_n \subseteq (a_n)$. Hence, 
        \begin{align*}
            -M < |\inf T_n| = |x_n|
        \end{align*}
        where $-M$ is an lower bound for $(x_n)$ then $(x_n)$ is bounded.

        Since $(x_n)$ is monotonic and bounded below, by the
         Monotone convergence \Thm $(x_n)$ converges.
    \end{proof}

    \item 
    Now $y_n = \sup T_n \geq \inf T_n = x_n$ so by the order limit \Thm, 
    $\lim y_n \geq \lim x_n$.

    The inequality becomes strict for sequence $(a_n) = \{ 1,-1,1,-1,... \}$ 
    since $\limsup a_n = 1 > \liminf a_n = -1$.

    \item 
    \begin{proof}
        $(\Rightarrow)$ Suppose $\liminf a_n = \limsup a_n = \alpha$. 
        For $\varepsilon > 0$ choose $N_1, N_2 \in \mathbb{N}$
        s.t. 
        \begin{gather*}
            |x_{n_1} - \alpha| < \varepsilon/3 \quad \forall n_1 \geq N_1\\
            |y_{n_2} - \alpha| < \varepsilon/3 \quad \forall n_1 \geq N_2
        \end{gather*}
        so
        \begin{align*}
            |x_n - y_n| &\leq |x_n - \alpha + \alpha - y_n| \\
            &\leq |x_n - \alpha| + |\alpha - y_n| \\
            &= 2\varepsilon/3
        \end{align*}
        Then take $N = \max\{ N_1, N_2 \}$ so that for all $n \geq N$
        \begin{align*}
            x_n \leq a_n \leq y_n
        \end{align*}
        and observe
        \begin{align*}
            |a_n - \alpha| &= |a_n - y_n + y_n - \alpha| \\
                            &\leq |a_n - y_n| + |y_n - \alpha| \\
                            &\leq |z_n - y_n| + |y_n - \alpha| \\
                            &< 2\varepsilon/3 + \varepsilon/3 \\
                            &= \varepsilon
        \end{align*}

        $(\Leftarrow)$ Suppose $(a_n) \rightarrow \alpha$, then 
        $\exists N \in \mathbb{N}$ s.t. 
        \begin{align*}
            |a_n - \alpha| < \varepsilon /2
        \end{align*} 
        for $n \geq N$ and this implies
        \begin{align*}
            \alpha - \varepsilon /2 < a_n < \alpha + \varepsilon /2
        \end{align*}
        so 
        \begin{gather*}
            x_n \geq x_N \geq \alpha - \varepsilon /2 \\
            y_n \leq y_N \leq \alpha + \varepsilon /2
        \end{gather*}
        where have used the facts that $x_n$ is monotonicity increasing and
        $y_n$ is monotonicity decreasing.
        So 
        \begin{align*}
            \alpha - \varepsilon /2 \leq x_N \leq x_n \leq a_n \leq y_n \leq y_N \leq \alpha + \varepsilon /2
        \end{align*}
        This chain of inequalities implies 
        \begin{gather*}
            \alpha - \varepsilon  < \alpha - \varepsilon /2 \leq x_n \leq \alpha + \varepsilon /2 < \alpha + \varepsilon \\
            \alpha - \varepsilon < \alpha - \varepsilon /2 \leq y_n \leq \alpha + \varepsilon /2 < \alpha + \varepsilon
        \end{gather*}
        which implies
        \begin{gather*}
            |x_n - \alpha| < \varepsilon \\ 
            |y_n - \alpha| < \varepsilon \\ 
        \end{gather*}
        so $(x_n)\rightarrow \alpha$ and $(y_n)\rightarrow \alpha$.
    \end{proof}
\end{enumerate}

