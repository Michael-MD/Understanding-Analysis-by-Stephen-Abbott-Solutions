\subsection{Consequences of Completeness}

\ex{1}
\begin{theorem}
    For every two real numbers a and b
    with $a<b$, there exists a rational number r satisfying $a<r<b$.
\end{theorem}
\begin{proof}
    We consider two cases:
    \begin{itemize}
        \item $a < 0, b > 0$:
        By \Thm 1.4.3 $\exists$ $r$ s.t.
        \begin{equation*}
            a < 0 \leq r < b
        \end{equation*}

        \item $a<0, b \leq 0$:
        By \Thm 1.4.3 $\exists$ $r$ s.t. $-b < r < -a$ 
        $\Rightarrow a < -r < b$.
    \end{itemize}
\end{proof}

\ex{2}
\begin{enumerate}[label=(\alph*)]
    \item 
    \begin{proof}
        Let $p/q, r/s \in \mathbb{Q}$ with $p, q, r, s \in \mathbb{Z}$, then
        \begin{align*}
            \frac{p}{q} + \frac{r}{s} = \frac{ps+rq}{qs} \in \mathbb{Q}
        \end{align*}

        Similarly,
        \begin{align*}
            \frac{p}{q} \frac{r}{s} = \frac{pq}{rs} \in \mathbb{Q}
        \end{align*}
    \end{proof}

    \item 
    \begin{proof}
        We proceed by contradiction:
        Suppose $p,q,r,s \in \mathbb{Z}$ with $a=p/q$, then
        \begin{align*}
            at = \frac{p}{q}t   &= \frac{r}{s} \\
                  \Rightarrow t &= \frac{rq}{sp} \in \mathbb{Q} 
        \end{align*}
    \end{proof}
    However, $t$ is irrational.

    \item 
    $\mathbb{I}$ is not closed under addition or multiplication.
    For instance, consider examples
    \begin{align*}
        - \sqrt 2 + \sqrt 2 = 0 \in \mathbb{Q} \\
        \sqrt 2 \sqrt 2 = 2 \in \mathbb{Q} 
    \end{align*}
\end{enumerate}

\ex{3}
\begin{corollary}
    Given any two real numbers $a<b$, there exists an irrational
    number $t$ satisfying $a<t<b$.
\end{corollary}
\begin{proof}
    Using \Thm 1.4.3 $\exists \, r \in \mathbb{Q}$ s.t. 
    \begin{equation*}
        a - \sqrt 2 < r < b - \sqrt 2 
    \end{equation*}
    given $a < b$. However, if we add $\sqrt 2$ this implies 
    \begin{equation*}
        a < r + \sqrt 2 < b
    \end{equation*}
    Using \Ex{2} we can conclude that $t = r + \sqrt 2 \in \mathbb{I}$.
\end{proof}

\ex{4}
\begin{proof}
    Let $A = \inf \{ 1/n: n \in \mathbb{N} \}$
    \begin{enumerate}[label=(\alph*)]
        \item 
        Since $1/n \neq 0$ for any $n \in  \mathbb{N}$ then 0
        is indeed a lower bound for A.

        \item 
        By the Archimedean Property of $\mathbb{R}$ for any
        $\varepsilon > 0 \, \exists n \in \mathbb{N}$ s.t.
        \begin{equation*}
            0 + \varepsilon = \varepsilon > 1/n \in A
        \end{equation*}
    \end{enumerate}
    Hence $\inf A = 0$.
\end{proof}

\ex{5}
\begin{proof}
    We proceed by contradiction, suppose $\exists x \in \bigcap_{n=1}^{\infty}$,
    However, by the Archimedean Property $ \exists m \in \mathbb{N}$ s.t.
    $x > 1/n$ so $x \notin (0,1/m) \Rightarrow x  \notin \bigcap_{n=1}^{\infty} (0,1/n)$
\end{proof}

\ex{6}
\begin{enumerate}[label=(\alph*)]
    \item 
    Continuing on from the proof of \Thm 1.4.5:
    By the Archimedean Property $\exists$ $n_0 \in \mathbb{N}$ s.t. 
    $1 / n_0 < \frac{\alpha^2+2}{2\alpha} $ which implies
    \begin{equation*}
        \left( \alpha - \frac{1}{n_0} \right) > \alpha^2 - \frac{2\alpha}{n_0} > 2
    \end{equation*}
    But we assumed $\alpha$ is a least upper bound so we have reached a 
    contradition. Hence, $\alpha^2 = 2 \Rightarrow \alpha=\sqrt 2$.

    \item
    We only need to repeat the proof given in \Thm 1.4.5 using the set
    \begin{align*}
        S = \{ s\in \mathbb{R}: s^2 < b \}
    \end{align*}
    We would conclude $\sup S = \sqrt b$.
\end{enumerate}

\ex{7}
\begin{theorem}
    If $A \subseteq B$ and $B$ is countable, then $A$ is either countable,
    finite, or empty.
\end{theorem}
\begin{proof}
    We proceed by induction: We will induct on the domain of $g: \mathbb{N} \rightarrow A$.
    
    \begin{itemize}
        \item 
        Base case ($m = 1$):
        Let $T_1 = \{ n_1 \in \mathbb{N} : f(n) \in A \}$. Let $n_1 = \min T_1$
        and $g(1) = f(n_1)$.

        \item 
        Inductive step: ($m > 1$): Let $T_{m+1} = T_m\\ n_m$. Let $n_{m+1} = \min T_{m+1}$
        and $g(n_{m+1}) = f(n_{m+1})$.
    \end{itemize}
    Clearly $g$ will be 1-1 and onto since $\text{dom} g$ and $\text{ran} g$ are 
    subsets of $\text{dom} f$ and $\text{ran} f$ respectively which is also 1-1 and onto.
    
    If $A$ is either empty or finite then the indcution stops at some $m$.
\end{proof}

\ex{8}
\begin{enumerate}[label=(\alph*)]
    \item 
   \begin{proof}
        Without loss of generality suppose $A_1, B_2$ are disjoint. We begin by
        assuming that both sets are infinite. Since $B_2 \subseteq A_2$ we proved
        in \Ex{7} that $B_2$ is countable so $\exists$ bijections
        \begin{align*}
            f &: A_2 \rightarrow \mathbb{R} \\
            g &: B_2 \rightarrow \mathbb{R}
        \end{align*}
        We define new function $h : \mathbb{N} \rightarrow A_1 \cup B_2$ by
        \begin{equation*}
            h(n) = \begin{cases} 
                f(\frac{n+1}{2}) & n \text{ odd} \\
                g(\frac{n}{2}) & n \text{ even}
            \end{cases}
        \end{equation*}
        $h$ simply returns the elements of $A_1$ and $B_2$ in alternating order.
        Since $h$ is 1-1 (since $A_1$ and $B_2$ are disjoint) and onto we can
        conclude $A_1 \cup B_2$ is countable. However since $A_1 \cup B_2 = A_1 \cup A_2$
        then the same conclusion follows for any two infinite countable sets.
        
        If $B_2$ is finite with $m$ elements then we simply define
        \begin{equation*}
            h(n) = \begin{cases} 
                g(n) & n \leq m \\
                f(n-m) & n > m
            \end{cases}
        \end{equation*}

        By induction this result can be generalized to an arbitrary number of sets
        since every induction step only involves the union of two countable sets
        which we have proven.
   \end{proof}

   \item
   We cannot induct on infinity, what case came before infinity?

   \item
   Let $f_m: \mathbb{N} \rightarrow A_m \, \forall m \in \mathbb{N}$ and 
   $A = \{ f_m \in m\in \mathbb{M} \}$.

   For each $m\in \mathbb{N}$, we could assign $A_m$ to the $m^{\text{th}}$ row
   of the grid and the $n^{\text{th}}$ element of $A_m$ to the $n^{\text{th}}$ 
   column. To that end, we define bijective function $k : \mathbb{N} \times \mathbb{N} \rightarrow A$ given by
   \begin{equation*}
        k(n,m) = f_n(m)
   \end{equation*}

   Next let $g: \mathbb{N} \times \mathbb{N} \rightarrow \mathbb{N}$ which 
   takes in a row and a column and returns the corresponding value in the 
   two-dimensional array in illustrated in the preblem. Hence $g$ is bijective 
   so $g^{-1}$ also exists and undoes the mapping.
   
   The composition of bijective functions is bijective so we can define
   $\ell : \mathbb{N} \rightarrow \bigcup_{n=1}^\infty A_n$ By
   \begin{equation*}
        \ell = k \circ g^{-1}
   \end{equation*}
   which would prove $\bigcup_{n=1}^\infty A_n$ is countable.
    
\end{enumerate}

\ex{9}
\begin{enumerate}[label=(\alph*)]
    \item 
    If $A \sim B$ then $\exists$ bijection $f: A\rightarrow B$. However, 
    the inverse of a bijection is a bijection so $f^{-1}$ exists and is bijective
    which implies $B\sim A$.

    \item
    The result follows trivially using the fact that the composition of 
    bijective functions is a bijection. That is, if $f:A \rightarrow B$ and 
    $g:B \rightarrow C$ then $h:A\rightarrow C$ is a bijection and implies $A \sim C$.
\end{enumerate}

\ex{10}
We first prove a lemma which will make the desired proof much simpler:
\begin{lemma}
    $\mathbb{N} \times \mathbb{N} \sim \mathbb{N}$
\end{lemma}
\begin{proof}
    Now
    \begin{equation*}
        \mathbb{N} \times \mathbb{N} = \bigcup_{n}^\infty \{ (n,m) : m \in \mathbb{N} \}
    \end{equation*}
    Each set in the union is clearly countable and we know the union of countable sets is countable.
\end{proof}

We can extend this lemma using induction to conclude that
\begin{align*}
    \underbrace{\mathbb{N} \times \cdots \times \mathbb{N}}_n \sim \mathbb{N}
\end{align*}
for any $n \in \mathbb{N}$.

We can now proceed with the desired proof.
\begin{proof}
    Let
    \begin{equation*}
        B_n = \{ \text{all subsets of } \mathbb{N} \text{ with n elements} \}
    \end{equation*}
    It is clear that $B_n \sim \underbrace{\mathbb{N} \times \cdots \times \mathbb{N}}_n$
    for any $n \in \mathbb{N}$ so $B_n$ is countable. We know the union of countable sets is
    countable so 
    \begin{equation*}
        \bigcup_n^\infty B_n \sim \mathbb{N}
    \end{equation*}
\end{proof}

\ex{11}
\begin{enumerate}[label=(\alph*)]
    \item 
    We can define map
    \begin{equation*}
        f : (0,1) \rightarrow S
    \end{equation*}
    where for $s\in $ $f(s) = (s,0)$

    \item 
    We can define $f:S \rightarrow (0,1)$ such that for $(x,y) \in S$ $f$ interleves the 
    digits of $x$ and $y$. For example if $x=0.321$ and $y=0.458$ then $f(x,y)=0.435281$.
    However, $f$ is not onto, for example, $\nexists s\in S$ s.t. $f(s)=0.101010...$.  
\end{enumerate}

\ex{12}
\begin{enumerate}[label=(\alph*)]
    \item
    \begin{itemize}
        \item $\sqrt 2$ is the \soln of $x^2-2=0$
        \item $\sqrt[\uproot{3} n]{2}$ is a \soln of $x^3 - 2=0$
        \item 
        Now
        \begin{align*}
            (\sqrt 3 + \sqrt 2)^2 &= 3 + 2 + 2 \sqrt 6 \\
            (\sqrt 3 + \sqrt 2)^2 - 5 &= 2 \sqrt 6 \\
            \Rightarrow ((\sqrt 3 + \sqrt 2)^2 - 5)^2 &= 24
        \end{align*}
        So $\sqrt 3 + \sqrt 2$ is a \soln of $(x^2-5)^2 = 24$.
    \end{itemize}

    \item
    We can divide the set of degree $n$ polynomials based on whether their coefficients 
    satisify
    \begin{equation*}
        |a_n| + |a_{n-1}| + \cdots + |a_1| + |a_0| \leq m
    \end{equation*}
    There are less than $2m^n$ unique polynomials, each with a finite number of roots so 
    the set of algebraic numbers obtained from polynomials of degree $n$ with terms 
    summing to less than or equal to $m$ is finite and denoted by $A_{(n,m)}$. 
    The union of finite sets is countable so $A_n = \bigcup_m^{\infty} A_{(n,m)}$ is countable
    where $A_n$ denotes the set of all algebraic numbers obtainable from polynomials of 
    degree $n$.

    Alternative method:
    A quick way to see this result is to note that 
    \begin{align*}
        \underbrace{\mathbb{Z} \times \cdots \times \mathbb{Z}}_{m} \sim \underbrace{\mathbb{N} \times \cdots \times \mathbb{N}}_{2m} \sim \mathbb{N}
    \end{align*}
    Since we can easily find a mapping of polynomials of integer coefficients of degree $n$ to 
    the above set which is countable, then by the transitive property set of polynomials of integer coefficients 
    of degree $n$ is countable. Each polynomials produces at most four roots and the union of
    a countablely infinite number of finite sets. So if we denote
    \begin{align*}
        B_m = \{\text{Set of roots of the } m^{\text{th}} \text{ polynomial}\}
    \end{align*}
    then $\bigcup_m^\infty B_m$ is countable.

    \item 
    Since the union of countable sets is countable $\bigcup_n^\infty A_n$ is countable.
\end{enumerate}

\ex{13}
\begin{enumerate}[label=(\alph*)]
    \item 
    $f$ is by assumption 1-1 so $\exists$ a unique $x\in X$ s.t. $f(x)=y$ given $y\in f(X)$.

    \item 
    \begin{itemize}
        \item 
        The number of elements to the left of $x$ is zero: 
        If $X=Y$ and $f, g$ are identity maps $g^{-1}(x)=x$ and $f^{1}(g^{-1}(x))=x$ etc. so
        there are no values to the left of $x$.

        Alternatively: $x\in g^{-1}(X)$.

        \item 
        The number of elements to the left of $x$ is finite: 
        If $g^{-1}(x) \not\in f^{-1}(X)$ then only one element exists to the left of $x$.

        \item 
        A simple example is if $g^{-1}(x) = y$ and $f^{-1}(y) = x$.
        
        A less trivial example can be obtained if we consider $X=Y=\mathbb{N}$ and 
        $f(n) = n+1 \rightarrow f^{-1}(n)=n-1$. Let $g=f$, then if we chain $f^{-1}$ and 
        $g^{-1}$ we obtain $..,n-2, n-1, n, ...$
    \end{itemize}

    \item 
    If $\exists$ some $a, b$ s.t. any one of the following holds
    \begin{gather*}
        a = (f^{-1} \circ \cdots \circ f^{-1})(b) \\
        a = (f^{-1} \circ \cdots \circ g^{-1})(b) \\
        a = ( g^{-1} \circ \cdots  \circ f^{-1})(b) \\
        a = ( g^{-1} \circ \cdots \circ g^{-1} )(b)
    \end{gather*}
    The $C_a$ and $C_b$ match at atleast one point. If this is the case then since
    the chain is extending in the same way and the starting value is the same then 
    the two chains must agree everywhere.

    Alternatively, if the above does not occur then the chains must not agree anywhere
    and are therefore disjoint.

    \item
    If $C_y \cap Y\not\subseteq f(X)$ then $\exists$ at least one a single $y\in Y$ where
    $y \not\in f(X)$. Hence, the chain cannot be extended to the left since $f^{-1}(y)$
    is undefined.

    \item 
    If $C_x \in  A$, then every $y\in A$ is in $f(X)$. This implies that every $x\in X$
    proceeding $y$ is in the doman of $f$.
    So $X_1 \subseteq \text{dom} f$ and $Y_1 \subseteq \text{ran} f$. Hence, since $f$ is
    by assumption 1-1 then $f: X_1 \rightarrow Y_1$ is onto also.

    Similarly, if $C_y \in  B$, then every $x\in B$ is in $g(Y)$ since every element in $x\in X$
    is proceeded by an element in $y$ with $g(y)=x$
    So $Y_2 \subseteq \text{dom} g$ and $X_2 \subseteq \text{ran} g$. Hence, since $g$ is
    by assumption 1-1 then $g: Y_2 \rightarrow X_2$ is onto also.

    Now since $f:X_1 \rightarrow Y_1$ and $g: Y_2 \rightarrow X_2$ are bijections $f^{-1}, g_{-1}$
    exist and are also bijections. 
    
    We are now ready to define a bijection $h: X \rightarrow Y$. Since
    \begin{align*}
        A &\text{ disjoint } B \\
       \Rightarrow X_1 &\text{ disjoint } X_2 \\
       \Rightarrow Y_1 &\text{ disjoint } Y_2
    \end{align*}
    $X_1 \cup X_2 = (A \cap X) \cup (B \cap X) = X \cap (A \cup B) = X$ since $A \cup B$ is the
    set of all possible chains. Similarly, $Y_1 \cup Y_2 = Y$.

    So we define 
    \begin{align*}
        h(x) = 
        \begin{cases}
            f(x) & x \in X_1 \\
            g^{-1}(x) & x \in X_2
        \end{cases}
    \end{align*}
    Since $X_1$ and $X_2$ are disjoint, $h$ is well-defined. In addition, since $X_1 \cup X_2 = X$
    and $Y_1 \cup Y_2 = Y$ $h$ is 1-1 and onto. Similarly,
    \begin{align*}
        h^{-1}(x) = 
        \begin{cases}
            f^{-1}(x) & x \in Y_1 \\
            g(x) & x \in Y_2
        \end{cases}
    \end{align*}
    Since we have found bijection between $X$ and $Y$, $X \sim Y$.

\end{enumerate}
