\subsection{The Axiom of Completeness}

\ex{1}
\begin{enumerate}[label=(\alph*)]

    \item
    For some $z \in \mathbb{Z}_5$, if we choose $y = 5 - z \pmod 5$ then
    \begin{align*}
        z + y   &= (5 - z) + z \pmod 5 \\
                &= 0 \pmod 5 
    \end{align*}
    Hence $y$ is an additive identity of $z$.

    \item
    This can be done explitly. \\
    \begin{center}
        \begin{tabular}{cc}
            \toprule
            $z$ & $x$ \\
            \midrule
            1 & 1 \\
            2 & 3 \\
            3 & 2 \\
            4 & 4 \\
            \bottomrule
        \end{tabular}
    \end{center}

    For example, for $z = 4, x = 4$:
    \begin{align*}
        xz  &= 4 \times 4 \pmod 5 \\
            &= 16 \pmod 5 \\
            &= 1 \pmod 5
    \end{align*}

    \item
    For $\mathbb{Z}_4$, then the construction in (a) will always be valid for a additive
    inverse. Hence an additive inverse exists for $\mathbb{Z}_n \, \forall \, n \in \mathbb N$. 

    Finding a multiplictive inverse is not always possible for any $z \in \mathbb Z_4$. For
    instance, $z = 2$ does not have a multiplictive inverse since any number multiplied
    by two will be even. However, we require the product to be $4n+1$ for some $n \in \mathbb N$
    which is impossible since the product will always be even.

    Hence, we conjecture that $\exists$ a multiplitive inverse $\forall \, z \in \mathbb Z_n$
    given $n$ is prime so that no element of $\mathbb Z_n$ shares a factor with n i.e. the greatest
    common divisor between any element of the set and n is 1.
    
\end{enumerate}

\ex{2}
\begin{enumerate}[label=(\alph*)]
    \item 
    \begin{definition}
        A real number $i$ is a greatest lower bound for a set $A \subseteq \mathbb{R}$ if it meets the following criteria:
        \begin{enumerate}[label=(\roman*)]
            \item $i$ is a lower bound for $A$.
            \item If $b$ is any lower bound for $A$, then $b \leq i $.
        \end{enumerate}
    \end{definition}

    \item
    \begin{lemma}
        Assume $i \in \mathbb{R}$ is a lower bound on a set $A \subseteq \mathbb{R}$. 
        Then $i = \inf A$ if and only if $\forall \, \varepsilon > 0 \, \exists \, a \in A $
        s.t. $i + \varepsilon > a$.  
    \end{lemma}

    \begin{proof}
        ($\Rightarrow$) Suppose $i = \inf A$, then $\forall \, \epsilon > 0 $ $ i+\varepsilon > i$
        but $i+\varepsilon$ cannot be a greater lower bound so there must exist $a \in A$ s.t. $i+\varepsilon > a$.

        ($\Leftarrow$) Suppose for some $i \in \mathbb{R}$ $i+\varepsilon > a$ for some $a \in A $ for arbitrary $ \varepsilon > 0$ then:
        \begin{enumerate}[label=(\roman*)]
            \item $i$ is a lower bound by assumption.
            \item Suppose $i \neq b = \inf A$ with $b>i$, then $b = i + \varepsilon$ for 
            $\varepsilon = |b - i|$ but $b =  i + \varepsilon > a$ for some $a \in A$ so
            we have reached a contradition since we assumed $b$ is a lower bound on A.
            Hence, $i$ is indeed the greatest lower bound.
        \end{enumerate}
    \end{proof}
\end{enumerate}

\ex{3}
\begin{enumerate}[label=(\alph*)]
    \item 
    \begin{proof}
        \begin{enumerate}[label=(\roman*)]
            \item 
            Suppose $\inf A$ is not an upper bound on B so $\exists$ $b \in B$ s.t. $\inf A \leq b$ but $b$ is a lower bound on $A$ so
            \begin{equation*}
                \inf A < b \leq a \, \forall a \in  A
            \end{equation*}
            This is a contradition since this would imply $b$ is a greater lower
            bound on A so we conclude $\inf A$ is indeed an upper bound on B.

            \item 
            A lower bound on $A$ is given by $\inf A - \varepsilon$ for any $\varepsilon > 0$ so $\inf A - \varepsilon \in B \forall \varepsilon > 0$. 
            But this would imply
            \begin{equation*}
                \inf A - \varepsilon < \inf A - \varepsilon/2 = b \in B
            \end{equation*}
            for any $\varepsilon >0$ which implies that $\inf A$ is a least upper bound for $B$.

            Alternatively, $\inf A$ is a lower bound on $A$ then $\inf A \in B$.
            So since $\inf A$ is an upper bound within $B$ then $\inf A$ is a maximum of $B$. If a maximum exists then $\sup B = \max B = \inf A$.
        \end{enumerate}
        Hence $\sup B = \inf A$.

        \item
        For set $A$ define set $-A = \{-a : a \in  A \}$. It is clear that $-A$ is bounded so we only need to prove
        \begin{equation*}
            -\sup(-A) = \inf A
        \end{equation*}
    \end{proof}
\end{enumerate}

\ex{4}
\begin{proof}
    If $B \subseteq A$ then $\forall b\in B \, \exists a \in  A$ s.t. $a \geq b$, this implies
    \begin{equation*}
        \sup A \geq a \geq \sup B
    \end{equation*}
\end{proof}

\ex{5}
\begin{enumerate}[label=(\alph*)]
    \item 
    \begin{proof}
        \begin{enumerate}[label=(\roman*)]
            \item 
            By \defn, $\forall a \in A \, \sup A \geq a$. 
            Adding $c$ to both sides we get $\forall a \in A \, c + \sup A \geq c + a $. Hence, $c + \sup{A}$ is an upper bound on $c+A$.

            \item 
            Now for arbitrary $\varepsilon > 0$, $\sup A - \varepsilon < a$ for some $a \in A$. This implies, $c + \sup A - \varepsilon < c + a \in c + A$ for some $a \in A$.
        \end{enumerate}
        Hence, $\sup{c+A} = c + \sup A$.
    \end{proof}

    \item
    \begin{proof}
        \begin{enumerate}[label=(\roman*)]
            \item 
            By \defn, $\forall a \in A \, \sup A \geq a$. 
            Multiplying by $c$ on both sides we get $\forall a \in A \, c \sup A \geq c a $. Hence, $c \sup{A}$ is an upper bound on $c A$.

            \item 
            Now for arbitrary $\varepsilon > 0$, $\sup A - \varepsilon < a$ for some $a \in A$. This implies, $c \sup A - \varepsilon < c a \in c A$ for some $a \in A$.

        \end{enumerate}
    \end{proof}

    \item 
    We postulate $\sup{cA} = -c \inf A$. This can be easily seen by imagining the sequence of maniplations from $A \rightarrow cA \rightarrow -cA$ and 
    observe how the up changes.
\end{enumerate}

\ex{6}
\begin{enumerate}[label=(\alph*)]
    \item $3$
    \item $1$
    \item $1/2$
    \item $9$
\end{enumerate}

\ex{7}
\begin{proof}
    Since is an upper bound by assumption, we only need to check that
    a is the least upepr bound. However, this is trivial since for arbitrary $\varepsilon > 0$ $a - \varepsilon < a \in A$ so $a = \sup A$.
\end{proof}

\ex{8}
\begin{proof}
    Let $\varepsilon = (\sup A + \sup B) / 2$. So
    \begin{equation*}
        \sup A < \sup B - \varepsilon < b
    \end{equation*}
    for some $b \in B$. So $b \in  B$ is an upper bound for $A$. 
\end{proof}

\ex{9}
\begin{enumerate}[label=(\alph*)]
    \item True
    
    \item 
    False, consider set 
    \begin{equation*}
        A = \left\{ L - \frac{1}{n} : n \in \mathbb{N} \right\}
    \end{equation*}
    where $\sup A = L$.
    
    \item 
    False, consider sets
    \begin{align*}
        A &= \left\{ L - \frac{1}{n} : n \in \mathbb{N} \right\} \\
        B &= \left\{ L + \frac{1}{n} : n \in \mathbb{N} \right\}
    \end{align*}
    However, $\inf B = \sup A = L$.

    \item True

    \item 
    False, consider sets
    \begin{align*}
        A &= \left\{ 1 - \frac{1}{n} : n \in \mathbb{N} \text{ and } n \text{ is even} \right\} \\
        B &= \left\{ 1 - \frac{1}{n} : n \in \mathbb{N} \text{ and } n \text{ is odd} \right\}
    \end{align*}
    Then, $\sup A = 1 \leq \sup B = 1$ but $\nexists \, b \in B$ that is an upper bound for $A$.

    This statement can be contradited more easily by noting that the 
    statement is false for two indentical sets with the supremum not in 
    the set. The above example demonstrates this statement is
    false for disjoint sets also.
\end{enumerate}
    
