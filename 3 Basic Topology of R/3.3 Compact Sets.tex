\subsection{Compact Sets}

\ex{1}
\begin{proof}
    Since $s=\sup K$, then we know 
    \begin{align*}
        s - 1/n < x_n \in K
    \end{align*}
    We can use this to construct a sequence where 
    $s - 1/n<x_n<s - 1/(n+1)$ and we know $(x_n)\rightarrow s$.
    Since $K$ is compact then it is closed so $s\in K$.

    Similarly, since $i=\inf K$, then we know 
    \begin{align*}
        i + 1/n > y_n \in K
    \end{align*}
    We can use this to construct a sequence where 
    $i + 1/n<y_n<i + 1/(n+1)$ and we know $(y_n)\rightarrow i$.
    Since $K$ is compact then it is closed so $i\in K$.
\end{proof}

\ex{2}
\begin{proof}
    Consider any set $(x_n)$ in $K$, since $K$ is bounded then 
    $(x_n)$ is bounded also so by the Bolzano-Weierstrass \Thm
    $(x_n)$ has a convergent subsequence $(x_{n_k}) \rightarrow x$.
    Since $K$ is closed then $x\in K$. So $K$ is compact. 
\end{proof}

\ex{3}
\begin{proof}
    We know
    \begin{align*}
        C = \left[ (\infty, 0) \cup (1/3,2/3) \cup \cdots \right]^c
    \end{align*}
    Since we know the union of open sets is open, then the complement must be closed.
    Additionally since the Cantor set is bounded by unity then $C$ must be 
    compact by the Heine-Borel \Thm.
\end{proof}

\ex{4}
\begin{proof}
    The intersection of closed sets is closed to $K \cap F$ is closed.
    
    $K\cap F \subseteq K$ so if $K$ is bounded then $K\cap F$ is bounded.
    
    Hence, $K\cap F$ is compact.
\end{proof}

\ex{5}
\begin{enumerate}[label=(\alph*)]
    \item $\mathbb{Q}$ is not compact since it is not bounded or closed.
    \item $\mathbb{Q}\cap[0,1]$ is now bounded but it is still not closed. For
    example,
    there exists a convergent sequence in $\mathbb{Q}\cap[0,1]$ whcih converges to 
    $\ln 2$ which is not contained in $\mathbb{Q}\cap[0,1]$. So $\mathbb{Q}\cap[0,1]$
    is not compact.

    \item $\mathbb{R}$ is not closed but not bounded so $\mathbb{R}$ is not comapct.
    
    \item $\mathbb{R}\cap[0,1]=[0,1]$ which is comapct.
    \item This set is not closed since this is a seqeunce which converges to zero
    which is not in the set. So this set is not comapct.
    \item This set is bounded by $1$ and closed so it is compact.
    
\end{enumerate}

\ex{6}
\begin{enumerate}[label=(\alph*)]
    \item 
    For $x_1, y_1 \in C$ s.t. $x_1+y_1=s$:
    
    \begin{align*}
        x_1=\begin{cases}
            s/2 & s/2 \in C_n \\
            s/2+1/3 & \text{otherwise} \\
        \end{cases}
    \end{align*}

    \begin{align*}
        y_1=\begin{cases}
            s/2 & s/2 \in C_n \\
            s/2-1/3 & \text{otherwise} \\
        \end{cases}
    \end{align*}

    Since the cantor set is a fractal, a 
    construction of $x_n$ and $y_n$ in $C_n$
     is as  follows:

    \begin{align*}
        x_n=\begin{cases}
            s/2 & s/2 \in C_n \\
            s/2+1/3^n & \text{otherwise} \\
        \end{cases}
    \end{align*}

    \begin{align*}
        y_n=\begin{cases}
            s/2 & s/2 \in C_n \\
            s/2-1/3^n & \text{otherwise} \\
        \end{cases}
    \end{align*}

    \item
    $(x_n)$ and $(y_n)$ are bounded sequences so by the 
    Bolzano-Weierstrass \Thm there exist convergent 
    subsequences $(x_{n_k}) \rightarrow x$ and
     $(y_{n_k})\rightarrow y$. 

     Since $C_{n+1} \subseteq C_{n}$ then if $x_{n_k}\in C_n$ then 
     $x_{n_k} \in C_m$ for $m \geq n_k$. For any $C_m$, there exists a 
     point in the subsequence $(x_{n_k})$ s.t. $x_{n_k}\in C_m$. 
     So the tail of $(x_{n_k})$ is in every $C_m$ and converges to $x$.
     Since $C_m$ is closed $x \in C_m$ for all $m$. A similar arguement applies for
     $(y_{n_k})$. Hence $x,y\in C$.
\end{enumerate}

\ex{7}
\begin{enumerate}[label=(\alph*)]
    \item 
    True, since the arbitrary intersection of closed sets is closed and 
    bounded since each set is bounded.

    \item 
    False, consider $A=(0,1)$ and $K=[0,1]$ then $A\cap K=A$ which is not
    compact.

    \item False, consider the sets $F_n=[n, \infty)$, then 
    \begin{align*}
        \bigcap_n F_n = \bigcap_n [n, \infty) = \emptyset
    \end{align*}

    \item True, since a finite set is always closed and bounded.
    \item False, $\mathbb{Q}$ is countable but not compact.
\end{enumerate}

\ex{8}
\begin{enumerate}[label=(\alph*)]
    \item 
    Since $K \subseteq (A_1\cap K) \cup (B_1\cap K)$ then if both had 
    finite subcovers then the union of finite subcovers of $(A_1\cap K)$
    and $(B_1\cap K$ would be a finite subcover of $K$ also.

    \item 
    Choose $I_1$ to be the interval of $I_0$ where the intersection 
    with $K$ has no finite subcover.
    $I_1$ can again be bisected into $A_2, B_2$, for the same reason as outlined in (a)
    $A_2\cap K$ or $ B_2\cap K$ or both do not have a finite subcover.This process is 
    repeated indefinitely to obtained nested intervals 
    \begin{align}
        I_0 \supseteq I_1 \supseteq I_2 \supseteq \cdots
    \end{align}
    By construction each of these sets cannot be finitely covered.
    Also, since $|I_n| = |I_0|/2^n$ then $\lim |I_n| = 0$.

    \item 
    Since $I_0, I_1, I_2, ...$ are all compact sets so $I_n \cap K$ is
    also compact. $I_n \cap K$ is nonempty for all $n$ since every interval 
    when intersected with $K$ has no finite subcover. If $I_n\cap K$ was 
    not empty then a finite subcover exists.
    
    Then by the NIP for compact sets 
    there exists $x \in \cap (I_n\cap K)$. So $x\in I_n\cap K$ for all $n$ 
    so $x\in I_n$ and $x\in K$.

    \item
    Since $x\in K$ then there must exist an open set $O_{\lambda_0}$
    where $x\in O_{\lambda_0}$. Within $O_{\lambda_0}$ there is an open interval 
    centered around $x$ with width $\varepsilon$. We know that there exists $N\in \mathbb{N}$
    such that $|I_n|< \varepsilon$ for $n\geq N$. However, we have reached a contradiction
    since we assumed every interval had no finite subcover, yet $O_{\lambda_0}$ is a
    cover. Hence, $K$ must have a finite subcover.
\end{enumerate}

\ex{9}
\begin{enumerate}[label=]
    \item (b)
    Consider $A=(0,1)$ and $K=[0,1]$ then $A\cap K=A$ which is not
    compact. Then 
    \begin{align*}
        \{O_n = (1/n,1) : n\in \mathbb{N}\}
    \end{align*}
    is an open cover of $A$ which has no finite subcover.

    \item (e)
    $\mathbb{Q}$ is a countable set yet
    \begin{align*}
        \{O_n = (-n,n) : n\in \mathbb{N}\}
    \end{align*}
    is an open cover which cannot be made finite while still covering $\mathbb{Q}$.
\end{enumerate}

\ex{10}
For any set a closed cover for set $S$ is
\begin{align*}
    \{ [x,x] : x\in S \}
\end{align*}
Hence, if $S$ is infinite then there cannot exist a finite closed subcover.

Conversely, suppose $S$ is finite. Then from any (potentially) infinite closed 
subcover ${F_\lambda}$, for every $s\in S$ choose $\lambda_s$ s.t. $s\in F_{\lambda_x}$.
This is will be a closed finite subcover.

Hence, a set is clompact if and only if it is finite.