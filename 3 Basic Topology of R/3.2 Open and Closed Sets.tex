\subsection{Open and Closed Sets}

\ex{1}
\begin{enumerate}[label=(\alph*)]
    \item 
    The proof assumes there exists $V_\varepsilon(a) \subseteq \bigcap O_k$.
    If we allow for an infinite number of intersections then the intersection 
    may be a singleton set which is closed $\nexists$ $V_\varepsilon(a) \subseteq \bigcap O_k$.
    
    \item
    Consider 
    \begin{align*}
        O = \bigcap (a-\frac{1}{n}, a+\frac{1}{n}) 
    \end{align*}
\end{enumerate}

\ex{2}
\begin{enumerate}[label=(\alph*)]
    \item 
    $b_n = \frac{(-1)^n n }{n+1}$ has two convergent subsequences, 
    these being 
    $(\frac{n }{n+1}) \rightarrow 1$ and $(\frac{-n}{n+1}) \rightarrow -1$.
    So the limit points are $\pm 1$.

    \item
    No, since it does not contain all of its limit points

    \item 
    No, since every point is an isolated point.

    \item
    Yes, every point is isolated since
    \begin{align*}
        V_\varepsilon(b_n) \bigcap B = \emptyset
    \end{align*}
    for $\varepsilon = \min\{ |b_n - b_{n+2}|, |b_n - b_{n-2}|, |b_n - b_{n-1}|, |b_n - b_{n+1}| \}$ i.e.
    the distance between terms surrounding $b_n$.

    \item
    $\bar B = B \cup \{\pm 1\}$
\end{enumerate}

\ex{3}
\begin{enumerate}[label=(\alph*)]
    \item 
    $\mathbb{Q}$ is not open since any neighbourhood
    of a rational point will contain an irrational number.
    $\mathbb{Q}$ is not closed since every irrational number is
    a limit point of $\mathbb{Q}$. These are all consequences
    of the density of $\mathbb{Q}$ i $\mathbb{R}$.

    \item 
    $\mathbb{N}$ is not open since every point is isolated.
    $\mathbb{N}$ is closed since it has no limit points 
    other than those corresponding to $\{n,n,n,n,n,...\}$.

    \item 
    This set is open since every element has an open neighbourhood
    contained in the set. It is not clsoed since the limit 
    $(1/n)$ is not in the set so it does not contain all of its limit 
    points.

    \item 
    This interval is not open or closed since the limit of $(1/n)$
    is not in the set and $\nexists V_\varepsilon(1) \subseteq \{ x\in \mathbb{R} : x>0 \}$.

    \item
    These are the partial sums of $(1/n^2)$, since $0$
    is not in the set then this set is not closed.
    It is not open since every point is an isolated point.
\end{enumerate}

\ex{4}
\begin{proof}
    Suppose $x=\lim a_n$ then for any $\varepsilon$-neighbourhood
    there exists $N\in \mathbb{N}$ s.t. 
    \begin{align*}
        |x-a_n| < \varepsilon \forall n \geq N
    \end{align*}
    with $a_n \neq x$ so $a_n \in V_\varepsilon(x)\cap A \forall n\geq N$.
    Since $\varepsilon$ was arbitrary then $x$ is a limit point.
\end{proof}

\ex{5}
\begin{proof}
    $(\Rightarrow)$ Suppose $a$ is an isolated point then by 
    definition $a$ is not a limit point so there exists some 
    $\varepsilon$-neighbourhood s.t. 
    \begin{align*}
        V_\varepsilon(a) \cap A = \{ a \}
    \end{align*}

    $(\Leftarrow)$ Suppose $V_\varepsilon(a)\cap A = \{ a \}$.
    Then $a$ is not a limit point since we have found an $\varepsilon$-neighbourhood
    with only $a$ inside so $a$ is an isolated point. 
\end{proof}

\ex{6}
\begin{proof}
    Suppose $F \subseteq \mathbb{R}$ is closed
    $\Leftrightarrow$ $F$ contains all its limit points $\{ x_n \}$
    $\Leftrightarrow$ for each limit point there exists a sequence 
    $(a_n)\in F$ with $a_n \neq x_n$ and $(a_n)\rightarrow x_n$
    $\Leftrightarrow$ $(a_n)$ is a cauchy sequence.
\end{proof}

\ex{7}
\begin{proof}
    Since $x \in O$ then there exists an $\varepsilon$-neighbourhood
     $V_\varepsilon(x)\subseteq O$. However, since $(a_n)\rightarrow x$
     then there exists $N\in \mathbb{N}$ s.t. $a_n \in V_\varepsilon(x) $
     for $n \geq N$. Since the tail of the sequence is infinite then the 
     set of terms not contained in $O$ i.e. $\{x_1, x_2, ..., x_{N-1}\}$
     is finite since it contains $N-1$ terms.
\end{proof}

\ex{8}
\begin{enumerate}[label=(\alph*)]
    \item 
    \begin{proof}
        Suppose $L$ is not closed so there exists 
        $(l_n) \rightarrow s$ with $l_i \in L$ and $s\not\in A$.
        For $\varepsilon>0$ there exists $l_n \in V_{\varepsilon/2}(s)$.

        However, since $l_n$ is a limit point of $A$ then there exists 
        $a_n \in  A$ s.t. $a_n \in V_{\varepsilon/2}(l_n)$.
        This implies
        \begin{align}
            |s-a_n| &= |s-l_n+l_n-a_n| \\
                    &\leq |s-l_n|+|l_n-a_n| \\
                    &< \varepsilon/2 + \varepsilon/2 \\
                    &=\varepsilon
        \end{align}
        If we choose $\varepsilon_n = 1/n$ we can make sequence
         $(a_n)$ which is in $A$ and converges to a limit point of 
         $L$ so every limit point of $L$ is a limit point of $A$ 
         which by defintiion is contained in $L$ so $L$ is closed.
    \end{proof}

    \item
    \begin{proof}
        If $x$ is a limit point of $A\cup L$ then either $x$ is a
        limit point of $A$ or $L$ or both. If it is a limit point of 
        both then there exists a subsequence only in $A$ or only in $L$
        which converges to the same value so we can regard $x$ as a limit point
        of $A$ or $L$. We previously showed if $x$ is a limit point of $L$
        then it is a limit point of $A$ so $x$ is a limit point of $A$ always.
        
        Any limit point of $A\cup L$ is also 
        a limit point of $A$ which is in $A\cup L$ so $A\cup L$ is closed. 
        $\bar A$ is the smallest possible set containing $A$ since we remove an element, $t$,  
        then if $t \in A$ then $A\nsubseteq \bar A$. If we remove a limit point then $\bar A$
        is not closed so this is the smallest possible set which achieves both.
    \end{proof}
\end{enumerate}

\ex{9}
\begin{enumerate}[label=(\alph*)]
    \item 
    If $y$ is a limit point of $A\cup B$ then there exists sequence $(y_n)\rightarrow y$
    where $y_n \in A\cup B$. If $(y_n)$ is eventually in $A$ or $B$ then $y$ is a limit point
    of $A$ or $B$ respectively. If $(y_n)$ si frequently in both then $A$ and $B$ have a limit point
    in common, this should not be misunderstood as $y$ is a limit of $A\cup B$.

    \item 
    So $\overline{A\cup B} = A\cup B \cup L_A \cup L_B \cup (L_{A\cap B}) = 
    \bar{A\cup B} = (A\cup L_A)  \cup ( B  \cup L_B) = \bar A \cup \bar B$.

    \item
    No, consider this counter example. 
    \begin{align*}
        \overline{\bigcup_n^\infty (\frac{1}{n}, 1)} = [0,1]
    \end{align*}
    However,
    \begin{align*}
        \overline{\bigcup_n^\infty (\frac{1}{n}, 1)} &\neq \bigcup_n^\infty \overline{(\frac{1}{n}, 1)} \\
                                                    &= \bigcup_n^\infty [\frac{1}{n}, 1] \\
                                                    &= (0,1]
    \end{align*}
\end{enumerate}

\ex{10}
\begin{enumerate}[label=(\alph*)]
    \item 
    \begin{proof}
        First we prove $(\cup E_\lambda)^c = \cap E_\lambda^c$.

        Suppose $x\in (\cup E_\lambda)^c$, then $x\not\in E_\lambda \forall \lambda$
        so $x \in \cap E_\lambda^c$. This implies $(\cup E_\lambda)^c \subseteq \cap E_\lambda^c$.

        Conversely, suppose $x\in \cap E_\lambda^c \Rightarrow x\in E_\lambda^c \forall \lambda$
         $\Rightarrow x\not\in E_\lambda \forall \lambda \Rightarrow x\not\in \cup E_\lambda$
         $\Rightarrow x\in (\cup E_\lambda)^c$. This implies $\cap E_\lambda^c \subseteq (\cup E_\lambda)^c$.

        Hence, $\cap E_\lambda^c = (\cup E_\lambda)^c$.

        We can apply this result to $\{F_\lambda = E_\lambda^c\}$ so 
        \begin{align*}
            \cap F_\lambda^c &= (\cup F_\lambda)^c \\
            \cap (E_\lambda^c)^c &= (\cup E_\lambda^c)^c \\
            \cap E_\lambda &= (\cup E_\lambda^c)^c
        \end{align*}
        Taking the complement we obtain
        \begin{align*}
            (\cap E_\lambda)^c &= \cup E_\lambda^c
        \end{align*}
        which is the second result we needed to prove.
    \end{proof}

    \item
    \begin{proof}
        From \Thm 3.2.3 we know that the union of an arbitrary number of 
        open sets is open i.e. $\cup O_\lambda$.
        We also now the complement of an open set is closed so applying De Morgan's
        laws we obtain
        \begin{align*}
            (\cup O_\lambda)^c &= \cap O_\lambda^c \\
            (\cap O_\lambda)^c &= \cup O_\lambda^c
        \end{align*} 
        The first line states the LHS is closed and the RHS is the intersection of an arbitrary number of 
        closed sets.

        The second line states the LHS is closed and the RHS is the union of a finite number of 
        closed sets.
    \end{proof}
\end{enumerate}

\ex{11}
\begin{proof}
    For $\varepsilon_n=1/n$
    \begin{align*}
        s-\varepsilon_n < a_n < s
    \end{align*}
    so $a_n \in V_{\varepsilon_n}(s)$ so we have found a sequence with 
    $(a_n) \rightarrow s$ and $a_n \in A \forall n\in \mathbb{N}$
    so $s \in \bar A$.
\end{proof}

\ex{12}
\begin{enumerate}[label=(\alph*)]
    \item True, the closure of a set is closed and the complement of a closed set is open.
    \item True, since there exists an open neighbourhood not contained in A.
    \item \begin{proof}
        $(\Rightarrow)$ Suppose $A$ is closed, then it contains all its limit points
        so $A=A\cup L=\bar A$.
        $(\Leftarrow)$ Suppose $A=\bar A$, then $A$ contains all its limit points so $A$ 
        is closed.
    \end{proof}
    \item True, we proved this in \Ex{11}.
    \item True, since every sequence in a finite set which converges must have a constant tail
    using an element of the set.
    \begin{proof}
        True. We can prove this by contradiction. Suppose there is an irrational number $y\in A$ not
        in the set so it is an isolated point i.e. $V_\varepsilon(y)\cap A = \{y\}$.
        However, the rationals are dense in $\mathbb{R}$ so there must exist a ration number 
        in $V_\varepsilon(y)$ and we have reached a contradiction.
    \end{proof}
\end{enumerate}

\ex{13}
\begin{proof}
    Suppose $S$ is both open and closed, then $S$ cannot be finite
    since every point is isolated and so $S$ is not open. So $S$ must be
    empty or infinite.

    Suppose $S$ is infinite, since $S$ is open then it can be written as the union of open intervals
    i.e.
    \begin{align*}
        S = \bigcup_i (x_i, y_{i})
    \end{align*}
    where $x_i, y_{i}$ are the start and end of the $i$-th interval.
    So $x_i, y_i \in S^c$ which is also open so there eixts $V_{\varepsilon_{x_i}}(x_i) \subseteq S$.
    However, any neighbourhood of $x_i$ contains elements of $S$ so $S^c=\emptyset$ so $S=\mathbb{R}$. 
\end{proof}

\ex{14}
\begin{enumerate}[label=(\alph*)]
    \item 
    \begin{align*}
        [a,b] = \cap (a-1/n, b+1/n)
    \end{align*}

    \item
    \begin{align*}
        (a,b] &= \cap (a, b+1/n) \\
        (a,b] &= \cup [a+1/n, b] 
    \end{align*}

    \item
    \begin{itemize}
        \item 
        Consider set $S$ which is the intersection of open sets given by
        \begin{align*}
            S = \cap_i (qi, q_{i+1})
        \end{align*}
        where $q_i \in \mathbb{Q}$. Since $\mathbb{Q}$ is countable then these sets are countable.
        
        Observe
        \begin{align*}
            \mathbb{Q} = (\cap_i (qi, q_{i+1}))^c \\
            &= =cap_i (qi, q_{i+1})^c
        \end{align*}
        so $\mathbb{Q}$ is an $F_\sigma$ set.

        \item
        \begin{align*}
            \mathbb{I} = \overline{\mathbb{Q}} = \cap_i (qi, q_{i+1})
        \end{align*}
        so $\mathbb{I}$ is an $G_\delta$ set.
    \end{itemize}
\end{enumerate}


