\subsection{The Intermediate Value Theorem}

\ex{1}

\ex{2}
\begin{enumerate}[label=(\alph*)]
    \item 
    False, consider $f:(0,1)\rightarrow \mathbb{R}$
    where $f(x)=1/x$. Then $f$ is continuous on a 
    bounded domain but $f(0,1)$ is unbounded.

    \item
    False, since consider $f:(0,2) \rightarrow \mathbb{R}$
    given by $f(x)=(x-1)^2$. It is clear $f$ is continuous 
    on $(0,2)$ which is bounded and open, yet the codomain 
    is given by $[0,1)$ which is not open.

    \item
    True, bounded closed sets are compact by the 
    Heine-Borel \Thm and continuous functions preserve 
    comapctness.
\end{enumerate}

\ex{3}
This is impossible since take compact set 
$K\subset \mathbb{R}$. Then $f(K)\subseteq \mathbb{Q}$
is compact but any subset of $\mathbb{Q}$ is not 
compact since it cannot be closed even if it were bounded 
so such a function cannot exist.

\ex{4}
\begin{proof}
    Suppose $c\in (a,b)$ then $a<c<b \Rightarrow f(a)\leq f(c)\leq f(b)$.

    Now for arbitrary $\varepsilon>0$ and without loss
    of generailty we assume $V_\varepsilon(f(c))\subset (f(a),f(b))$
    \begin{align*}
        b_l = \max\{f(c)-\varepsilon/2, a\} \leq f(c) \leq
         b_u=\min\{f(c)+\varepsilon/2,b\}
    \end{align*}

    By IVP of $f$ $\exists x\in [a,c)$ s.t. $f(x)=b_l$ 
    (If $b_l=f(a)$ then $x=a$).
    Similarly, $\exists y\in (c,b]$ s.t. $f(y)=b_u$. So
    \begin{align*}
        f(a) \leq f(x)\leq f(c)\leq f(y) \leq f(b)
    \end{align*}
    $\Rightarrow a<x<c<y<b$ so choose $\delta$ s.t. 
    \begin{align*}
        V_\delta(c) \subset (x,y)
    \end{align*}
    Since $f$ is increasing this implies 
    \begin{align*}
        f(V_\delta(c)) \subseteq V_{\varepsilon/2}(f(c)) \subset
         V_{\varepsilon}(f(c))
    \end{align*}

    If $c=a$ then 
    \begin{align*}
        f(c) \leq b_u=f(c)+\varepsilon/2 \leq f(b)
    \end{align*}
    By IVP of $f$ $\exists y\in (a,b)$ s.t. $f(y)=b_u$ so  
    \begin{align*}
        f(c)\leq f(y)\leq f(b)
    \end{align*}
    $\Rightarrow c<y<b$. Next choose $\delta$ s.t. 
    \begin{align*}
        V_\delta(c) \subset [a,y)
    \end{align*}
    and the rest of the proof proceeds in the same way.

    An analgous arguement can be used to conclude $f$ is
    continuous at $b$. 
    
    Hence, $f$ is continuous on $[a,b]$.
\end{proof}

\ex{5}
\begin{proof}
    If we wish to prove the IVT for $h$ which is continuous on 
    $(a,b)$ then to simplify the proof we define 
    function $f(x)=h(x) - L$ so that $f(a)<0<f(b)$.
    We need to show $\exists c$ s.t. $f(c)=0$.

    We begin by arguing $K$ is closed.
    For some convergent sequence $(x_n)\in K$ with 
    $\lim x_n = x$,
    by the order limit \Thm we know $f(x_n)\leq 0$ for all 
    $n$ which implies $f(x)\leq 0$ so $x\in K$.
    Hence, $c\in K$ so $f(c)\leq 0$.
    
    Consider seqeunce $(y_n)\rightarrow c$ where $c<y_n<b$ 
    (which is always possible since 
    $b\neq c$ since $b\not\in K$). Now since $f$ is 
    continuous
    $f(y_n)\geq 0$ implies $f(\lim y)=f(c)\geq 0$.
    Since $f(c)\geq 0$ and $f(c)\leq 0$ then $f(c)=0$
    so $f(c)=h(c)-L=0 \Rightarrow h(c)=L$.
\end{proof}

\ex{6}
\begin{proof}
    Consider $(a_n), (b_n)$ where $I_n=[a_n,b_n]$.
    Both sequences are monotone and bounded so both converge.
    Since $I_n$ are nested then by the NIP $\exists x\in \cap_n I_n$.
    For any $\varepsilon>0$ there exists $N$ s.t. $|I_n|<\varepsilon/2$ 
    for $n\geq N$ which implies 
    \begin{gather*}
        |a_n-x| < \varepsilon \\
        |b_n-x| < \varepsilon
    \end{gather*}
    for $n\geq N$. Hence, $(a_n)\rightarrow x$ and $(b_n)\rightarrow x$.
    Now $f(a_n)\leq 0$ and $f(a_n)\geq 0$ $\forall n$. Hence, 
    $f(x)\leq 0$ and $f(x)\geq 0$ which implies $f(x)=0$. 
\end{proof}

\ex{7}
\begin{proof}
    Let's define function $h(x)=f(x)-x$ and we are tasked with finding 
    a root to this function. 
    Now $h(0)=f(0)\geq 0$ and $h(1)=f(1)-1\leq 0$.
    \begin{itemize}
        \item Suppose $h(0)>0, h(1)<0$: Then by IVT there exists 
        $c\in [0,1]$ s.t. $h(c)=0$.

        \item $h(0)=0$: Then $h(0)=0$.
        \item $h(1)=1$: Then $h(1)=0$
    \end{itemize}
\end{proof}