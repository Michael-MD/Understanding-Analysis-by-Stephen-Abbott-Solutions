\subsection{Derivatives and the Intermediate Value Property}

\ex{1}
\begin{enumerate}[label=(\roman*)]
    \item 
    \begin{proof}
        We will use the sequential criterion for function limits:
        For any sequence $(x_n)\rightarrow c$ with everything in 
        $A$ then 
        \begin{align*}
            & \lim_n \frac{f(x_n)+g(x_n)-f(c)-g(c)}{x_n-c} \\
            &= \lim_n \frac{f(x_n)-f(c)}{x_n-c} + \frac{g(x_n)-g(c)}{x_n-c} \\
            &= \lim_n \frac{f(x_n)-f(c)}{x_n-c} + \lim_n \frac{g(x_n)-g(c)}{x_n-c} \\
            &= f'(c) + g'(c)
        \end{align*}
        where the third equailty follows by the algebraic limit theorem and 
        the assumption that $f'(c)$ and $g'(c)$ exist.
    \end{proof}

    \item
    \begin{proof}
        This time we will use the function limit definition:
        \begin{align*}
            \lim \frac{kf(x)-kf(c)}{x-c} &= k\lim \frac{f(x)-f(c)}{x-c} \\
                                        &= kf'(c)
        \end{align*}
    \end{proof}
\end{enumerate}

\ex{2}
\begin{enumerate}[label=(\alph*)]
    \item 
    \begin{align*}
        f'(c) = \lim \frac{\frac{1}{x}-\frac{1}{c}}{x-c} &= \lim \frac{c-x}{xc(x-c)} \\
        &= \frac{-1}{c} \lim \frac{1}{x} \\
        &= \frac{-1}{c^2}
    \end{align*}

    \item
    \begin{proof}
        Since $(1/x)'=-1/x^2$ 
        
        Let $h(x)=1/x$ so $(f h\circ g)(x)=f(x)/g(x)$.
        Using the chain rule
        \begin{align*}
            (f h\circ g)'(c) &= f'(c) (h\circ g)(c)  + f(c)(h\circ g)'(c)  \\ 
                             &= \frac{ f'(c)}{g(x)}  + f(c)[h'(g(c))g'(c)]  \\
                             &=  \frac{ f'(c)}{g(x)}  + f(c)\frac{-g'(c)}{g(c)^2}  \\
                             &=  \frac{f'(c)g(c) - f(c)g'(c)}{g(c)^2}
        \end{align*}
    \end{proof}

    \item
    \begin{proof}
        \begin{align*}
            (f(x)/g(x))'(c) &= \lim \frac{f(x)/g(x) - f(c)/g(c)}{x-c} \\
                        &= \lim \frac{f(x)g(c)-g(x)f(c)}{g(x)g(c)(x-c)} \\
                        &= \lim \frac{f(x)g(c)-f(c)g(c)+f(c)g(c)-g(x)f(c)}{g(x)g(c)(x-c)} \\
                        &= \lim \frac{g(c)(f(x)-f(c))+f(c)(g(c)-g(x))}{g(x)g(c)(x-c)} \\
                        &= \lim \frac{1}{g(c)^2}[\frac{g(c)^2}{g(x)}\frac{f(x)-f(c)}{x-c}+f(c)\frac{g(x)-g(c)}{x-c}\frac{g(c)}{g(x)}] \\
                        &= \lim \frac{1}{g(c)^2}[f'(c)g(c)+g(c)g'(c)]
        \end{align*}
        where we have used the fact that $g'(c), f'(c)$ exists and $g(x)$ is continuous at $c$
        so that $\lim g(x)=g(c)$.
    \end{proof}
\end{enumerate}

\ex{3}
Now 
\begin{align*}
    h'(0) = \lim \frac{h(x)-h(0)}{x} = \lim \frac{h(x)}{x}
\end{align*}
Observe, for any ration sequence which converges to zero 
\begin{align*}
    \lim \frac{h(q_n)}{q_n} = \lim \frac{q_n}{q_n} = 1
\end{align*}

For any irration sequence which converges to zero 
\begin{align*}
    \lim \frac{h(i_n)}{i_n} = \lim \frac{i_n}{i_n} = 0
\end{align*}

Inspired by this we contruct our own function 
\begin{align*}
    g(x) = \begin{cases}
        x^2 & x\in \mathbb{Q} \\
        0 & x\not\in \mathbb{Q}
    \end{cases}
\end{align*}
Observe, for any ration sequence which converges to zero 
\begin{align*}
    \lim \frac{g(q_n)}{q_n} = \lim \frac{q_n^2}{q_n} = 0
\end{align*}

For any irration sequence which converges to zero 
\begin{align*}
    \lim \frac{h(i_n)}{i_n} = \lim \frac{i_n^2}{i_n} = 0
\end{align*}
so the derivative exists at $x=0$ while still not existing 
anywhere else since $g$ remains discontinuous.


\ex{4}
\begin{enumerate}[label=(\alph*)]
    \item 
    We require 
    \begin{align*}
        |f_a(x)-f_a(0)| = |f_a(x)| < \varepsilon
    \end{align*}
    which implies $|x^a| < \varepsilon$ so we require $a>0$.

    \item
    Consider sequences $(x_n)\leq 0$ and $(y_n)\geq 0$ then 
    \begin{gather*}
        \frac{f_a(x_n)-f_a(0)}{x_n} = \frac{f_a(x_n)}{x_n} = 0 \\
        \frac{f_a(y_n)-f_a(0)}{y_n} = \frac{f_a(y_n)}{y_n} = \frac{1}{y_n^{1-a}}
    \end{gather*}
    When we take the limit we wish for both these sequences to converge to zero.
    For $\frac{1}{y_n^{1-a}}$ to converge to zero we require $1-a<0$
    which implies $a>1$.

    For this set of values, we now check if $f_a$ is continuous.
    We can now be confident $f_a$ is differentiable everywhere since zero 
    is where the function changes.

    Hence
    \begin{align*}
        f_a'(c) &= \begin{cases}
            ax^{a-1} & x\geq 0 \\
            0 & x < 0
        \end{cases} \\ 
        &= a\begin{cases}
            x^{a-1} & x\geq 0 \\
            0 & x < 0
        \end{cases}
    \end{align*}
    from (a) we know the function will be continuous at zero 
    if the exponent is greater than 1 so $a-1>0 \Rightarrow a>1$.

    \item
    From (b) we know we require the exponent to be greater than one so 
    $a-1>1 \Rightarrow a>2$.
\end{enumerate}

\ex{5}
\begin{enumerate}[label=(\alph*)]
    \item 
    The dilema is at zero so this is where we focus:
    \begin{align*}
        \lim \frac{x^a \sin(1/x)}{x} = \lim x^{a-1} \sin(1/x)
    \end{align*}
    We need the function to be approaching zero as it gets closer to the 
    origin so we require $a-1>0 \Rightarrow a>1$ for $g_a$ to be differentiable
    everywhere.

    Now 
    \begin{align*}
        g_a'(x) &= ax^{a-1}\sin(1/x)+x^a(-1/x^2)\cos(1/x) \\
                &= ax^{a-1}\sin(1/x)-x^{a-1}\cos(1/x) \\
    \end{align*}
    If we want this function to be unbounded on $[0,1]$ then
    simply we require
    \begin{gather*}
        a-1<0 \quad a-2<0 \\
        a<1    \quad a<2
    \end{gather*} 
    So we want to preserve differentiability on $\mathbb{R}$ so we choose $1<a<2$.

    \item
    For differentiability of $g_a$ we require $a>1$.
    For $g_a'$ to be continuous we require 
    \begin{gather*}
        a-1>0 \text{ and } a-2>0 \\
        a>1 \text{ and } a>2
    \end{gather*}
    to tame the oscillations.

    Hence,
    \begin{align*}
        g_a'(0) &= \lim \frac{x^a \sin(1/x)}{x} \\
                &= x^{a-1} \sin(1/x) \\
                &= 0
    \end{align*}
    So
    \begin{align*}
        g_a''(0) &= \frac{g'(0)-0}{x} \\
                &= x^{a-2} \sin(1/x) - x^{a-3} \cos(1/x)
    \end{align*}
    For this limit to exist we require 
    \begin{gather*}
        a-2>0 \text{ and } a-3>0 \\
        a>2 \text{ and } a>3
    \end{gather*}
    So to be discontinuous we require $a\leq 2$ or $a\leq 3$.
    so for $g_a$ to be differentiabe on $\mathbb{R}$ with 
    $g'$ continuous but not differentiabe at zero we require $2<x\leq 3$.

    \item
    For $g_a$ to be differentiabe we require $a>1$.

    For $g_a'$ to be differentiabe we require $a-2>0$ and $a-3>0 \Rightarrow$  
    $a>2$ and $a>3$ or simply $a>3$.

    Now 
    \begin{align*}
        g_a''(0) &= \lim \frac{g'(x)-g'(0)}{x} \\
                &= \lim \frac{g_a'(x)}{x} \\
                &= \lim - x^{-3+a} \cos{\left(\frac{1}{x}\right)} + a x^{-2+a} \sin{\left(\frac{1}{x}\right)} \\
                &= 0
    \end{align*}
    so
    \begin{align*}
        g_a''(x) &= 2 x^{-3+a} \cos{\left(\frac{1}{x}\right)} \\
        &\quad - 2a x^{-3+a} \cos{\left(\frac{1}{x}\right)} \\
        &\quad - x^{-4+a} \sin{\left(\frac{1}{x}\right)} \\
        &\quad - a x^{-2+a} \sin{\left(\frac{1}{x}\right)} \\
        &\quad + a^{2} x^{-2+a} \sin{\left(\frac{1}{x}\right)}
        \end{align*}
        
        For this not to be continuous at zero we need to the oscilations to remain constant 
        or grow so we require 
        \begin{gather*}
            a-3\leq 0 \text{ or } a-4\leq 0 \\ 
            a\leq 3 \text{ or } a\leq 4
        \end{gather*}
        so we choose $3<x\leq 4$.
\end{enumerate}

\ex{6}
\begin{enumerate}[label=(\alph*)]
    \item 
    Consdier sequences $(x_n)\rightarrow a$ and $(y_n)\rightarrow b$
    where $(x_n),(y_n)\in (a,b)$.
    Now since $g'(a)$ exists then 
    \begin{align*}
        \lim \frac{g(x)-g(a)}{x-a}
    \end{align*}
    exists. Since  $g'(a)<0$ then there exists $N\in \mathbb{N}$ s.t.
    \begin{align*}
        \frac{g(x_n)-g(a)}{x_n-a} < 0
    \end{align*}
    for all $n\geq N$.
    This implies 
    \begin{align*}
        g(a) > g(x_n)
    \end{align*}
    for all $n\geq N$.

    Similarly, since $g'(b)$ exists then 
    \begin{align*}
        \lim \frac{g(x)-g(b)}{x-b}
    \end{align*}
    exists. Since  $g'(b)>0$ then there exists $N\in \mathbb{N}$ s.t.
    \begin{align*}
        \frac{g(y_n)-g(b)}{y_n-b} > 0
    \end{align*}
    for all $n\geq N$.
    This implies 
    \begin{align*}
        g(b) > g(y_n)
    \end{align*}
    for all $n\geq N$. Note, since $y_n-b<0$ we flip the inequailty.

    \item
    \begin{proof}
        $g$ si continous which implies there exists $x_0,x_1 \in [a,b]$
        s.t. $g(x_0)\leq g(x)\leq g(b)$ for all $x\in [a,b]$ by the 
        extreme value \Thm. Since in (a) we proved $g(a),g(b)$ cannot 
        be the minimum then $x_0\neq a,b$ so $x\in (a,b)$.
        By the Interior Extremum \Thm/Fermat's \Thm we know $g'(x_0)=0$
        i.e. $g'(x_0) = f'(x_0)-\alpha=0$ which implies $f'(x_0)=\alpha$. 
    \end{proof}
\end{enumerate}

\ex{7}
\begin{enumerate}[label=(\alph*)]
    \item 
    A function is differentiabe at $c$ given 
    \begin{align*}
        \lim \frac{f(x)-f(c)}{x-c}
    \end{align*}
    exists. If this limit exists then for every $\varepsilon>0$
    \begin{align*}
        |x-c|<\delta \Rightarrow |f'(c)-\frac{f(x)-f(c)}{x-c}| <\varepsilon
    \end{align*}
    Motivated by this an equivilent definition is given by:
    \begin{definition}
        A function is differentiabe at $c$ iff 
        \begin{align*}
            \lim |\frac{f(c+h)-f(c)-h f'(c)}{h}|=0
        \end{align*}
    \end{definition}
    This definition lends itself nicely to defining 
    uniform differentiability.
    \begin{definition}
        A function is uniformly differentiabe if given 
        $\varepsilon>0$ there exists $\delta>0$ s.t. 
        \begin{align*}
            0<|h|<\delta \Rightarrow |\frac{f(c+h)-f(c)-h f'(c)}{h}|<\varepsilon
        \end{align*}
        for all $c$.
    \end{definition}

    Simply put, a uniformly differentiabe function is one in which
    for any point $c$, the 
    gradient will be within $\varepsilon$ of the true value 
    given the slope calculated from two points is within $\delta>0$.

    \item
    $f(x)=x$ is an example of a uniformly continuous function.

    \item
    Differentiable functions on a compact set are not necessarily 
    uniformly differentiabe. Take for instance,
    \begin{align*}
        g_2(x) = x^2 \sin(1/x)
    \end{align*}
    which is differentiable everywhere but the derivative is 
    not continuous at zero. 

    It can be shown that  a differentible function is uniformly differentiable iff
    it has a continuous derivative. Hence, $g_2$ is not uniformly differentiabe.
    
    However, we can also check this geometrically.
    We know $g_2'$ is not continuous at zero. No matter 
    how small $\delta>0$ is made, since the oscillations bunch 
    up at zero, the gradient through two points will always 
    vary from $-1$ to $1$ since 
    \begin{align*}
        g_2'(x)=-\cos\left(\frac{1}{x}\right) + 2x\sin\left(\frac{1}{x}\right)
    \end{align*} 
    This is due the $\cos$ in the expression.
\end{enumerate}

\ex{8}
\begin{enumerate}[label=(\alph*)]
    \item True, since by Darboux's \Thm the derivative possesses 
    the IVP so if $f'(a) \neq f'(b)$ then there exists $\alpha \in \mathbb{I}$
    s.t. $f'(a)< \alpha < f'(b)$ and $x\in (a,b)$ s.t. $f(x)=\alpha$.

    \item
    False, we proved this true for contonious functions but a derivative 
    does not need to be continuous.

    Take for example,
    \begin{align}
        f(x)=x/2 + x^2 \sin(1/x)
    \end{align}
    We know the derivative will have oscillations of magnitude greater than 
    or equal to unity and the oscillations will get packed closer together
    nearer the origin. So no matter how small $\delta>0$ is made at least 
    one cycle oof oscillations will be present within $\delta$ from 
    the origin. The offset by $x/2$ is to offset the derivative by 
    $1/2$ so that the oscillations dip below the x-axis. This value is 
    not unique.

    \item
    \begin{proof}
        Now since $f'(0)$ exists then $\exists \delta_1 >0 $ s.t. 
        \begin{align}
            0<|x|<\delta_1 \Rightarrow |\frac{f(x)-f(0)}{x}| < \varepsilon/2
        \end{align}
        Since $\lim_{x\rightarrow 0} f'(x)=L$ then 
        there exists  $\exists \delta_2 >0 $ s.t. 
        \begin{align}
            0<|x|<\delta_2 \Rightarrow |f'(x)-L| < \varepsilon/2
        \end{align}
        
        We need to show 
        \begin{align*}
            L=\lim \frac{f(x)-f(c)}{x-c}
        \end{align*}

        Let $\delta = \min\{\delta_1,\delta_2\}$ then for $0<|x|<\delta$
        \begin{align}
            |f'(x) - f'(0)| &\leq |f'(x) - f(x)| + |f(x) - f'(0)| \\
                            &\leq |f'(x) - f(x)| + |f(x) - f'(0)| \\
        \end{align}
    \end{proof}
\end{enumerate}