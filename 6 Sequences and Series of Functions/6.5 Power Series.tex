\subsection{Power Series}

\ex{1}
\begin{enumerate}[label=(\alph*)]
    \item 
    Observe $g(1)=\sum (-1)^{n+1}/n$ so the series 
    is abolutely convergent at $|x|<|x_0|$.
    This implies the series is uniformly convergent on 
    any closed subset which implies the convergence is 
    uniform on $(-1,1)$.

    Since the terms of a power series are continuous on $(-1,1)$
    then the limiting function $g$ is continuous on $(-1,1)$ also.

    We know the series converges for $x\in (-1,1]$. However, 
    observe $g(-1) = - \sum 1/n$ which diverges so the series is 
    not convergent on $[-1,1]$. Hence, the radius of convergence 
    must be $R=1$ so the series cannot converge for $|x|>1$.

    \item
    $g'$ is cerrtainly defined on $(-1,1)$.

    However
    \begin{align*}
        g'(1) = \sum (-1)^{n+1}
    \end{align*}
    diverges and 
    \begin{align*}
        g'(-1) = \sum (-1)^{n+1} (-1)^{n-1} = \sum (-1)^{2n}
    \end{align*}
    which also diverges so $g'$ is defined for $x\in(-1,1)$ only.
\end{enumerate}

\ex{2}
\begin{enumerate}[label=(\alph*)]
    \item 
    Consider
    \begin{align*}
        \sum x^n/n^2
    \end{align*}

    \item
    Consider
    \begin{align*}
        \sum (-1)^{n+1} x^n/n
    \end{align*}
    which converges at $1$ but diverges at $-1$. Furthermore,
    the convergence at $1$ is conditional since $\sum x^n/n$ diverges at unity.
    
    This implies 
    \begin{align*}
        \sum (-1)^{n+1} (-x)^n/n
    \end{align*}
    converges conditionally at $-1$ but diverges at $1$.

    \item
    Consider
    \begin{align*}
        \sum (-1)^{n} x^{2n}/n
    \end{align*}
    which converges at $\pm 1$. However,
    the convergence at $\pm1$ is conditional since $\sum 1/n$ diverges.

    \item
    Absolute convergence at $1$ implies 
    \begin{align*}
        \sum |a_n|
    \end{align*}
    converges. Then ther series at $-1$ corresponds to 
    \begin{align*}
        \sum a_n (-1)^n
    \end{align*}
    However, taking the absolute value of this we obtain
    \begin{align*}
        \sum |a_n|
    \end{align*}
    which we have already concluded converges so such a request is 
    impossible.
\end{enumerate}

\ex{3}
We now that every power series a radius of convergence. 
Within which the convergence is absolute. However, 
at the boundary i.e. $x=R$ the series may converge 
or diverge possibily conditionally. Hence, 
the only places a power series can converge conditionally 
are $x=\pm R$.

\ex{4}
\begin{enumerate}[label=(\alph*)]
    \item 
    \begin{proof}
        A power series converging on $(-R,R)$ 
        $\Rightarrow$ power series converges at $x_1 \in (-R,R)$ 
        $\Rightarrow$ power series converges absolutely at $|x_0| < |x_1|$
        by \Thm 6.5.1
        $\Rightarrow$ power series converges uniformly on $[-|x_0|,|x_0|]$
        by \Thm 6.5.2

        Since $f_n(x)=a_n x^n$ is continuous on $[-|x_0|,|x_0|]$ then 
        $f$ is continuous on $[-|x_0|,|x_0|]$.

        For any $x_0\in(-R,R)$ choose $x_0 < x_1 < R$ and apply the proceeding arguement
        which implies the series converges to a continuous function on the entire 
        radius of convergence.
    \end{proof}

    \item
    \begin{proof}
        A power series converging on $x=R$
        $\Rightarrow$ power series converges absolutely at $[0,R]$
        by Abel's \Thm
        We can apply  \Thm 6.5.1 and  \Thm 6.5.2 to conclude the series is convrgent 
        on $(-R,R)$ which implies it is convergent on $(-R,R]$.
    \end{proof}
\end{enumerate}

\ex{5}
\begin{proof}
    Since $a_n x^n \leq |a_n x_0^n|$ by assumption then $\sum |a_n x_0^n|$ 
    converges so by M-test $\sum a_n x^n$ converges uniformly on $[-|x_0|,|x_0|]$. 
\end{proof}

\ex{6}
\begin{proof}
    Now a compact set $K$ contains it's maximum $M$ and minimum $m$. 
    Since we are assuming the convergence 
    is pointwise at $M$ then by Abel's \Thm the series converges uniformly on $[0,M]$.

    Since the series converges at $M$ then the convergence is absolute for $x\in (-M,M)$
    by \Thm 6.5.1 and the convergence is uniform for any closed subinterval therein by 
    \Thm 6.5.2. Hence convergence is uniform on $(-M,M)$. So we know the series 
    must converge at $(-M,M]$. 

    By a similar arguement we can conclude the series converges at $[-m,m)$. Hence, 
    the series converges at $[-m,M]$. Since it must be the case that $K\subseteq [-m,M]$
    then the power series converges on compact set $K$.
\end{proof}

\ex{7}
\begin{proof}
    \begin{enumerate}[label=(\alph*)]
        \item 
        Consider series $\sum n s^{n-1}$. We can check if this series 
        converges using the ratio test.
        \begin{align*}
            \lim |\frac{(n+1)s^n}{s^{n-1}}| &= \lim |\frac{(n+1)s^n}{s}| \\
            &= \lim s(1+\frac{1}{n}) \\
            &= s < 1
        \end{align*}
        Hence, by the ratio test this series converges. However, we know 
        every convergent series requires the terms to go to zero i.e. $ns^{n_1}\rightarrow 0$.
        Hence, $ns^{n-1}$ is bounded.
    
        \item
        If we assume the series $\sum a_n x^n$ converges with radius of convergence 
        $R$. Fix $x\in (-R,R)$ choose some $t\in (-R,R)$ where $|x|<t<R$ for which 
        we know the series converges.

        Observe, for some $|x|<t<R$
        \begin{align*}
            |n a_n x^{n-1}| &= \frac{1}{t} (n |\frac{x^{n-1}}{t^{n-1}}|)|a_n t^n|
        \end{align*}
        Also observe 
        \begin{align*}
            |\frac{x^{n-1}}{t^{n-1}}| = |\frac{x}{t}|^{n_1}
        \end{align*}
        so let $s=|x/t|<1$.
        
        From (a) we know $ns^{n-1}$ is bounded i.e. $|a_n t^n| \leq M$
        for some $M>0$.
        So 
        \begin{align*}
            |n a_n x^{n-1}| &= \frac{1}{t} (n s^{n-1})|a_n t^n| \\
                        &\leq \frac{M}{t}  |a_n t^n|
        \end{align*}

        We know $\sum a_n t^n$ converges absolutely so $\sum |a_n t^n|$ converges. 
        By the M-test this implies $\sum n a_n x^{n-1}$ converges uniformly on
        $[-|x|,|x|]$. Since $x$ was arbitrary we can conclude the convergence is uniform on 
        $(-R,R)$. 

        However, uniform convergence of the derivitive and in knowing there exists at least 
        one point in $(-R,R)$ where the original series converges implies 
        $f'(x)=\sum n a_n x^{n-1}$ where $f(x) = \sum a_n x^n$ on $(-R,R)$.

        Furthermore, by \Thm 6.5.5 we know at the covnergence of $\sum n a_n x^{n-1}$ is uniform 
        on any compact set contained in $(-R,R)$.
    \end{enumerate}
\end{proof}

\ex{8}
\begin{enumerate}[label=(\alph*)]
    \item 
    \begin{proof}
        We know by the ratio test a series will converge if the limit of the ratio 
        is less than one. That is, we require 
        \begin{align*}
            \lim |\frac{a_{n+1}x^{n+1}}{a_nx^n}| = \lim |\frac{a_{n+1}}{a_n}||x| = L|x| < 1
        \end{align*}
        This implies $|x|<1/L$ or $x \in (-1/L,1/L)$. 
    \end{proof}

    \item
    \begin{proof}
        Again, by the ratio test
        \begin{align*}
            L|x| = 0
        \end{align*}
        so by the ratio test $|x|$ does not matter for the series to converge so the series 
        converges on all of $\mathbb{R}$.
    \end{proof}

    \item
    
    
    \item

\end{enumerate}

\ex{9}
\begin{proof}
    When $x=0$ we immediately see $a_0 = b_0$. Differentiating we obtain
    \begin{align*}
        \sum_{n=1} a_n n x^{n-1} = \sum_{n=1} b_n n x^{n-1}
    \end{align*}
    Again, when $x=0$ we get $a_n=b_1$.

    We can then use induction to conclude $a_n=b_n$ for all $n$.
\end{proof}

\ex{10}

\ex{11}

\ex{12}