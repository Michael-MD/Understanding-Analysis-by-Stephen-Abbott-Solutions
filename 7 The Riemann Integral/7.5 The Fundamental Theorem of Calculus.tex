\subsection{The Fundamental Theorem of Calculus}

\ex{1}
A continuous function is integrable so 
\begin{align*}
    G(x) = \int_a^x g
\end{align*}
is well-defiend and by FTC $G'=g$ so $g$ is the 
the derivitive of some function $G$.

\ex{2}
\begin{enumerate}[label=(\alph*)]
    \item 
    $f(x)$ is continous so it has an antiderivative so 
    \begin{align*}
        F(x) &= \begin{cases}
            \int_{-1}^x -x & x\leq 0 \\
            \int_{-1}^x x & x> 0
        \end{cases} \\
        &= \begin{cases}
            -x^2/2|_{-1}^x & x\leq 0 \\
            x^2/2|_{-1}^x & x> 0
        \end{cases} \\
        &= \begin{cases}
            -(x^2/2-1/2) & x\leq 0 \\
            x^2/2-1/2 & x> 0
        \end{cases}
    \end{align*}
    $F$ will be continuous everywhere as long as $f$ is 
    integrable. However, $F$ is differentiable when 
    $f$ is continuous which is everywhere. Furthermore, 
    $F'=f'$.

    \item
    $f(x)$ is continous so it has an antiderivative so 
    \begin{align*}
        F(x) &= \begin{cases}
           \int_{-1}^x 1 & x < 0 \\
           \int_{-1}^x 2 & x \geq 0
        \end{cases} \\
        &= \begin{cases}
            1-x & x < 0 \\
            1+2x & x \geq 0
        \end{cases}
    \end{align*}
    $F$ will be continuous everywhere as long as $f$ is 
    integrable. However, $F$ is differentiable when 
    $f$ is continuous which is everywhere except $x=0$. 
    Furthermore, 
    $F'=f'$ for $x\neq 0$.
\end{enumerate}

\ex{3}
To use the MVT we only require $F$ be differentiable on $(a,b)$,
not $[a,b]$. However, we still require $F$ to be continuous on 
$[a,b]$.
So we require $F'=f$ on $(a,b)$ and $F$ be continuous on $[a,b]$.

\ex{4}
\begin{enumerate}[label=(\alph*)]
    \item 
    \begin{align*}
        H(a) = \int_1^1 \frac{1}{t} dt = 0
    \end{align*}
    Since $1/t$ is continuous on $(0,\infty)$ then $H'(x)=1/x$
    for $x>0$.

    \item
    \begin{proof}
        H is differentiable so $H$ is continuous. Using the MVT 
        on $[x,y]$ where $x,y\in (0,\infty)$ with $x<y$ so 
        there exists $t\in(x,y)$ s.t. 
        \begin{align*}
            \frac{H(y)-H(x)}{y-x} = f'(t)>0
        \end{align*}
        which implies $H(y)-H(x)>0$ or $H(y)>H(x)$.
    \end{proof}

    \item
    \begin{proof}
        By the chain rule $g(x) = H(cx) = 1/x = H'(x)$ so 
        \begin{align*}
            H(x) &= \int_1^x g'(t) dt \\ 
            &= g(x) - g(1) \\
            &= H(cx) - H(x) \\
            &\Rightarrow H(cx) = H(c) + H(x) 
        \end{align*}
    \end{proof}
\end{enumerate}

\ex{5}
\begin{proof}
    $f_n'\rightarrow g$ uniformly and $f_n'$ are assumed to be 
    continous so $g$ is also continuous which implies $g$ is 
    integrable so 
    \begin{align*}
        G(x) = \int_a^x g + f(a)
    \end{align*}
    is well-defined.
    Furthermore, since $g$ is continuous $G$ is differentiable 
    and $G'=g$.

    Fix $x\in [a,b]$ then 
    \begin{align*}
        |G(x)-f(x)| &= |\int_a^x g - \int_a^x f_n' + \int_a^x f_n' - f(x) + f(a)| \\
        &= |\int_a^x (g - f_n') + f_n(x) - f_n(a) - f(x) + f(a)| \\
        &\leq \int_a^x |g - f_n'| + |f_n(x) - f(x)| + |f_n(a) - f(a)|
    \end{align*}
    Since $f_n'\rightarrow g$ uniformly, choose $N_1$ s.t. 
    \begin{align*}
        |g(t) - f_n'(t)| < \frac{\varepsilon}{3}
    \end{align*}
    where $t\in [a,x]$.

    Since $f_n\rightarrow f$ pointwise, choose $N_2$ s.t. 
    \begin{align*}
        |f_n(x) - f(x)| < \frac{\varepsilon}{3}
    \end{align*}
    and choose $N_3$ s.t. 
    \begin{align*}
        |f_n(a) - f(a)| < \frac{\varepsilon}{3}
    \end{align*}
    Hence, for $N=\max\{N_1, N_2, N_3\}$
    \begin{align*}
        |G(x)-f(x)| < \varepsilon
    \end{align*}
    so $G(x) = f(x)$ which implies $G'(x)=f'(x)$ which implies $g(x)=f'(x)$.
    Since $x$ was arbitrary then this is true for all $x\in [a,b]$.
\end{proof}

\ex{6}
\begin{proof}
    Assuming $f$ is continuous then $G = \int_a^x g$ is well-defined and
     $G'=g$. Since $G'=F'=f$ then $G(x) = F(x) + k$ so 
    \begin{align*}
        \int_a^b g = G(b) - G(a) = F(b) - F(a)
    \end{align*}
    which is (i) of FTC. 
\end{proof}

\ex{7}
\begin{proof}
    Assuming $f$ is continuous then $G = \int_a^x g$ is well-defined and
    $G'=g$ and $G$ is continuous so by MVT there exists $c\in(a,b)$ s.t 
    \begin{align*}
        \frac{G(b)-G(a)}{b-a} = G'(c) = g(c)
    \end{align*}
    However, $G(b)-G(a) = \int_a^b g$  and $ G'(c) = g(c)$ so
    \begin{align*}
        g(c) = \frac{\int_a^b g}{b-a}
    \end{align*} 
\end{proof}

\ex{8}
\begin{enumerate}[label=(\alph*)]
    \item 
    \begin{proof}
        \begin{align*}
            Vf &= \sup\{ \sum_k |f(x_k) - f(x_{k-1})|\} \\
            &= \sup\{ \sum_k |\int_{x_{k-1}}^{x_k} f'|\} \\ 
            &\leq \sup\{ \sum_k \int_{x_{k-1}}^{x_k} |f'|\} \\ 
            &= \sup\{ \int_{a}^{b} |f'|\} \\ 
        \end{align*}
        where the second equaily holds since $f'$ is assumed continuous
        so $f(x)=\int_a^x f'$.
    \end{proof}

    \item
    \begin{proof}
        Since $\int f'$ is continuous and differentiable then we can apply 
        the MVT on $[a,b]$ to obtain
        \begin{align*}
            \frac{\int_a^b f'}{b-a} = f'(t)
        \end{align*}
        for some $t\in(a,b)$. For a given partition, for every subitnerval 
        there exists $t_k\in(x_{{k-1}},x_k)$ s.t 
        \begin{align*}
            \frac{f(x_k)-f(x_{k-1})}{x_k-x_{{k-1}}} = f'(t)
        \end{align*}
        Taking absolute values and rearranging
        \begin{align*}
            |f(x_k)-f(x_{k-1})| = |x_k-x_{{k-1}}||f'(t)|
        \end{align*}
        so 
        \begin{align*}
            Vf &= \sup\{ \sum_k |f'(t_k)||x_k-x_{{k-1}}|\}
        \end{align*}
        Now $\int |f'| = L(|f'|)$ so 
        \begin{align*}
            \int |f'| &= \sup\{ \sum_k \inf\{ |f'(x_k-x_{{k-1}})| \}(x_k-x_{{k-1}})\} \\
            &\leq \sup\{ \sum_k |f'(x_k-x_{{k-1}})| (x_k-x_{{k-1}})\} \\
            &= Vf
        \end{align*}
        so $\int |f'|\leq Vf$ and in (a) we showed $Vf \leq \int |f'|$ so $Vf = \int |f'|$.
    \end{proof}
\end{enumerate}

\ex{9}
\begin{proof}
    \begin{align*}
        H'(1) &= \lim_{x\rightarrow 1} \frac{H(x)-H(1)}{x-1} \\
        &= \lim_{x\rightarrow 1} \frac{\int_1^x h}{x-1} \\
        &= \lim_{x\rightarrow 1} \frac{x-1}{x-1} \\
        &= 1
    \end{align*}
    Notice, $H'(1)\neq h(1)$ since $h$ is not continuous at $x=1$ so FTC does not 
    gauarantee this anymore.
\end{proof}

\ex{10}
\begin{proof}
    For contradition, suppose $F$ is differentiable at $x=c$ then 
    we need to find $F'(c)$ which is given by 
    \begin{align*}
        F'(c) &= \lim_{x\rightarrow c} \frac{F(x)-F(c)}{x-c} \\
        &= \lim_{x\rightarrow c} \frac{\int_c^x f}{x-c}
    \end{align*}
    If we assume $f$ is continuous then $F' = f$ so using L'hopital's 
    rule we obtain
    \begin{align*}
        F'(c) = \lim_{x\rightarrow c} f(x)
    \end{align*} 
    For this limit to exist we require both one-sided limits to exist which is 
    false so $F'(c)$ does not exist.
\end{proof}

\ex{11}
