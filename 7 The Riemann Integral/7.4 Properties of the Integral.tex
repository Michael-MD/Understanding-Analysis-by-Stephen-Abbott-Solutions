\subsection{Properties of the Integral}

\ex{1}
\begin{enumerate}[label=(\alph*)]
    \item 
    \begin{proof}
        We break up this proof into the following
        three cases
        \begin{itemize}
            \item $f(A)<0$: $M'=\sup |f|=-\inf f=-m$ $m'=\inf |f|=-\sup f=-M$.
            Hence, $M-m=M'-m'$.

            \item $f(A)>0$: $m'=\inf |f|=\inf f=m$ $M'=\sup |f|=\sup f=M$.
            Hence, $M-m=M'-m'$.

            \item $f(A)$ contains positive and negative elements:
            $m'=\inf |f| > \inf f=m$ $M'=\sup |f|=\sup f=M$.
            Hence, $M-m=M'-m>M'-m'$.
        \end{itemize}
    \end{proof}

    \item
    \begin{proof}
        Since $f$ is integrable, for $\varepsilon>0$ 
        choose parition $P$ s.t. 
        $U(f, P)-L(f, P) < \varepsilon$. This implies 
        \begin{align*}
            U(|f|, P)-L(|f|, P) &= \sum_j (M_j'-m_j')\delta x_j \\
            &\leq \sum_j (M_j-m_j)\delta x_j \\
            &= U(f, P)-L(f, P) \\
            &< \varepsilon
        \end{align*}
    \end{proof}

    \item
    \begin{proof}
        $f$ is integrable and $-|f|\leq f\leq |f|$ $\Rightarrow$ \Thm 7.4.2 (ii)
        and (iv)
        \begin{align*}
            - \int |f| \leq \int f \leq \int |f|
        \end{align*}
        and by \Thm 7.4.2 (iii) $\int |f|\geq 0$ so 
        \begin{align*}
            |\leq \int f| \leq \int |f|
        \end{align*}
    \end{proof}
\end{enumerate}

\ex{2}
\begin{proof}
    \begin{align*}
        \int_a^c+\int_c^b = -\int_c^a + \int_c^a + \int_a^b = \int_a^b
    \end{align*}
\end{proof}

\ex{3}
\begin{proof}
    From Exercise 7.2.5 we know uniform Convergence of a sequence 
    preserves integrability so $\int f$ is well-defined.
    We only need to establish the equailty $\lim_n \int f_n = \int f$.

    \begin{align*}
        |\int f - \lim_n \int f_n| &= |\lim_n \int (f -  f_n)| \\
        &= \lim_n |\int (f-f_n)| \\
        &\leq \lim_n \int |f-f_n|
    \end{align*}
    where the second equailty holds since $|\cdot|$ is differentiable.

    We know $f_n\rightarrow f$ uniformly so there exists $N$
    s.t. 
    \begin{align*}
        |f-f_n| < \frac{\varepsilon}{(b-a)}
    \end{align*}
    Hence for the same $N$
    \begin{align*}
        \lim_n \int |f-f_n| \leq \frac{\varepsilon}{(b-a)}(b-a) = \varepsilon
    \end{align*}
    by \Thm 7.4.2 (iii).
    Hence, 
    \begin{align*}
        |\int f - \lim_n \int f_n| < \varepsilon
    \end{align*}
    for $n\geq N$ which implies $\lim \int f_n = \int f$ as required.
\end{proof}

\ex{4}
\begin{enumerate}[label=(\alph*)]
    \item 
    False, consider
    \begin{align*}
        f(x) = \begin{cases}
            1 & x\in \mathbb{Q} \\
            0 & x\not\in \mathbb{Q} 
        \end{cases}
    \end{align*}
    which is not integrable, however, $|f(x)|=1$ is 
    certainly integrable.

    \item
    Consider,
    \begin{align*}
        g(x) = \begin{cases}
            1 & n\in \mathbb{N} \\
            0 & n\not\in \mathbb{N}
        \end{cases}
    \end{align*}
    Then this integrates to zero despite there being an (countabely) infinite 
    number of points where $g(x)>0$.

    An example, where the integration is not defined on the entire real 
    line is given by $g: [0,1]\rightarrow \mathbb{R}$
    \begin{align*}
        g(x) = \begin{cases}
            1 & x = 1/n : n\in \mathbb{N} \\
            0 & \text{otherwise}
        \end{cases}
    \end{align*}

    \item
    \begin{proof}
        We previously proved that for a continuous function $g$, if 
        $g(x_0)>0$ then there exists $\delta>0$ s.t. 
        \begin{align*}
            g(x)>0
        \end{align*}
        given $|x-x_0|<\delta$. Say $x_1 \in V_\delta(x_0)$ with 
        $x_0 < x_1$ then 
        \begin{align*}
            g(x)>0 \quad x\in [x_0,x_1]
        \end{align*}
        Hence, 
        \begin{align*}
            \int_a^b g = \int_a^{x_0} g + \int_{x_0}^{x_1} g + \int_{x_1}^{b} g 
        \end{align*}
        Now $ \int_a^{x_0} g\geq 0, \int_{x_1}^{b} g\geq 0$ and $\int_{x_0}^{x_1} g>0$
        so $\int_a^b g>0$.
    \end{proof}

    \item
    \begin{proof}
        Assume for contradition every interval $[c,d]\subseteq [a,b]$
        $f(x) \leq 0$. This implies $\sup(f(x))\leq 0$ and 
        $\inf(f(x))\leq 0$ which implies $\int f \leq 0$. This is a 
        contradition and the required result follows.
    \end{proof}
\end{enumerate}

\ex{5}
\begin{enumerate}[label=(\alph*)]
    \item 

\end{enumerate}

\ex{6}

\ex{7}
\begin{proof}
    Observe on $[0,1]$
    \begin{align*}
        |\lim_n \int_0^1 g_n - \int_0^1 g| &= |\lim_n \int_0^1 g_n-g| \\
        &= |\lim_n [\int_0^\delta (g_n-g) + \int_\delta^1 (g_n-g)]| \\
        &\leq |\lim_n \int_0^\delta (g_n-g) | + |\lim_n \int_\delta^1 (g_n-g)|
    \end{align*}
    On $[\delta, 1]$ we know $\lim_n \int g_n = \int g$ so we can find
    $N \in \mathbb{N}$ s.t. 
    \begin{align*}
        |\lim_n \int_\delta^1 (g_n-g)| < \varepsilon/2
    \end{align*}
    for $n\geq N$.

    As for $|\lim_n \int_0^\delta (g_n-g) |$, we know $g_n, g$ are 
    bounded by say $M>0$ so $|g_n-g|\leq 2M$.
    By \Thm 7.4.2 we know this implies  
    \begin{align*}
        |\lim_n \int_0^\delta (g_n-g) | \leq 2M\delta
    \end{align*}
    So if we choose
    \begin{align*}
        \delta < \frac{\varepsilon}{4M}
    \end{align*}
    we find 
    \begin{align*}
        |\lim_n \int_0^1 g_n - \int_0^1 g| &\leq  |\lim_n \int_0^\delta (g_n-g) | + |\lim_n \int_\delta^1 (g_n-g)| \\
        &<  2M\delta  + \varepsilon/2 \\
        &<  \varepsilon
    \end{align*}
\end{proof}