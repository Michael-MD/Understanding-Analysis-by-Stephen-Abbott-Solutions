\subsection{Integrating Functions with Discontinuities}

\ex{1}
\begin{enumerate}[label=(\alph*)]
    \item 
    For every partition the final subinterval will be of the form 
    $[x_n,1]$. However the infimum on this set of $h$ will always be 
    1.

    \item
    Consider partition $x_0=0, x_1=1-\frac{1}{20},x_2=1$, then 
    \begin{align*}
        U(h,P) &= (1-1/20)+(1/20)2 \\
        &= 1 + 1/20 < 1+1/10
    \end{align*}

    \item
    Consider partition $x_0=0, x_1=1-\frac{\varepsilon}{/2},x_2=1$, then 
    \begin{align*}
        U(h,P) &= (1-\varepsilon/2)+(\varepsilon/2)2 \\
        &= 1 + \varepsilon/2 < 1+\varepsilon
    \end{align*}
\end{enumerate}

\ex{2}
For some enumeration of the rationals $\{r_1, r_2, \dots\}$, consider 
\begin{align*}
    g_n(x) = \begin{cases}
        1 & x\in \{r_1, r_2, \dots, r_n\} \\
        0 & \text{otherwise}
    \end{cases}
\end{align*}

$g_n(x)$ is integrable since it has a finite number of discontinuities
but the limit which is dirichlet's function is not. This is expected since 
in Exercise 7.2.5 we proved that the limit is guaranteed to be 
integrable (given the terms of the sequence are integrable)
if the convergence is uniform. 

\ex{3}
\begin{proof}
    Choose the open interval so that all added the size of all the 
    intervals is $\varepsilon/8M$ where $M$ is the bound on $f$.

    Let $\{p_i\}_{i=1}^k$ be the set of discontinuities and 
    $\delta_i$ the size of each neighbourhood of point $p_i$ i.e.
    \begin{align*}
        \sum_i 2\delta_i = \frac{\varepsilon}{4M}
    \end{align*}
    Construct partitions $\{P_i\}$ where $P_i$ is a partiion 
    of interval $[p_i+\delta_i, p_{i+1}-\delta_{i+1}]$ where the 
    parition is chosen to satisfy 
    \begin{align*}
        U(f, P_i) - L(f,P_i) < \frac{\varepsilon}{2(k+1)}
    \end{align*}
    where $k$ is the number of discontinuities. The first and last 
    paritions will be the interval from the boundaries and the 
    first and last disconuity respectively.
    We know this is always possible since $f$ with the discontinuities 
    removed is continuous and therefore integrable.

    Now let $P = \cup_i P_i$, since $P$ is a refinement the above 
    inequailities will continue to be satisfied.
    
    Hence, 
    \begin{gather*}
        U(f, P) = \sum_i^{k+1} U(f, P_i) + \sum_{i=1}^k \sup f(V_{\delta_i}(p_i))2\delta_i \\
        L(f, P) = \sum_i^{k+1} L(f, P_i) + \sum_{i=1}^k \inf f(V_{\delta_i}(p_i))2\delta_i
    \end{gather*}
    so 
    \begin{align*}
        U(f, P) -  L(f, P) &= \sum_i^{k+1} U(f, P_i)  - L(f, P_i)  \\ 
        &+ \sum_{i=1}^k 2\delta_i (\sup f(V_{\delta_i}(p_i)) - \inf f(V_{\delta_i}(p_i)))
    \end{align*}
    Since $f$ is bouned $\sup f(V_{\delta_i}(p_i)) - \inf f(V_{\delta_i}(p_i)) \leq 2M$.
    So 
    \begin{align*}
        U(f, P) -  L(f, P) &\leq \sum_i^{k+1} U(f, P_i)  - L(f, P_i) + \sum 2\delta_i 2M \\
        &\leq \sum_i^{k+1} U(f, P_i)  - L(f, P_i) + 4M \sum \delta_i \\
        &< \frac{\varepsilon}{2(k+1)} \sum_i^{k+1} 1 + 4M \frac{\varepsilon}{8M} \\
        &= \varepsilon/2 + \varepsilon/2 \\
        &= \varepsilon 
    \end{align*}
\end{proof}

We could have proceeded with the proof by using the continuity of 
$f$ on a compact set which implies the set is now uniformly continuous. 
We could then argue that $U(f,P_i) - L(f,P_i)$ can be made 
arbitrarily small by paritioning $P_i$ s.t. each subinterval is less than 
$\delta>0$ where $|x-y|<\delta \Rightarrow |f(x)-f(y)|<\frac{\varepsilon}{2(b-a)}$.
However, this is les elegant I think.

\ex{4}
\begin{enumerate}[label=(\alph*)]
    \item 
    \begin{proof}
        We will denote the modified function by $g$.

        Let's parition the interval $[a,b]$ into three subintervals 
        $[a,p-\delta], [p-\delta, p+\delta], [p+\delta,b]$ where 
        $p$ is the point in the domain of $f$ which has been changed 
        and $\delta>0$ is some neighbourhood of $p$ to be determined.

        We can compose a partition of $[a,b]$ as the union of the paritions
        $[a,p-\delta], [p+\delta,b]$ which we will denote by $P_{[a,p-\delta]}, 
        P_{[p+\delta,b]}$ respectively and the refinement by $P= P_{[a,p-\delta]} 
        \cup P_{[p+\delta,b]}$, then 
        \begin{align*}
            U(g, P) - L(g, P) &= U(f, P_{[a,p-\delta]}) - L(f, P_{[a,p-\delta]}) \\ 
            &+ U(f, P_{[p+\delta,b]}) - L(f, P_{[p+\delta,b]}) \\
            &+ 2\delta(\sup g([p-\delta, p+\delta]) - \inf g([p-\delta, p+\delta]) ) \\
            &\leq U(f, P_{[a,p-\delta]}) - L(f, P_{[a,p-\delta]}) 
            + U(f, P_{[p+\delta,b]}) - L(f, P_{[p+\delta,b]}) 
            + 2\delta (2M)
        \end{align*}
        where $M$  is an upper bound on $g$ since $f$ is integrable and $g$ and $f$
        differ at a single point.
        Hence, we see if we choose $\delta = \frac{\varepsilon}{8M}$ and 
        partitions such that $U - L < \varepsilon/4$ then 
        \begin{align*}
            U(g, P) - L(g, P) &< \varepsilon
        \end{align*}
        Hence, $g$ remains integrable.

        To show that the integral remains unchanged it suffices to only
        show $U(f)=U(g)$. In addition, it suffices to show for $\varepsilon>0$
        there exists a partition such that 
        \begin{align*}
            U(f,P) < U(g)+ \varepsilon
        \end{align*}
        Observe,
        \begin{align*}
            U(f,P) &= \sum_i \sup f(\Delta x_i) \Delta x_i \\
            &= \sum_i \sup g(\Delta x_i) \Delta x_i + \sup f(\Delta x_j) \Delta x_j \\
            &\leq U(g, P) - \sup g(\Delta x_j) \Delta x_j + \sup f(\Delta x_j) \Delta x_j \\
            &\leq U(g, P) + \Delta x_j (\sup f(\Delta x_j) - \sup g(\Delta x_j))
        \end{align*}
        $g$ is integrable so if we choose a partition s.t. 
        \begin{align*}
            U(g, P) < U(g) + \varepsilon/2
        \end{align*}
        so 
        \begin{align*}
            U(g, P) \Delta x_j (\sup f(\Delta x_j) - \sup g(\Delta x_j)) < U(g) + \frac{\varepsilon}{2} + \Delta x_j (\sup f(\Delta x_j) - \sup g(\Delta x_j))
        \end{align*}
        We can refine the parition such that each subinterval is less than $\frac{\varepsilon}{2(2M)}$ 
        which implies 
        \begin{align*}
            & U(g) + \frac{\varepsilon}{2} + \Delta x_j (\sup f(\Delta x_j) - \sup g(\Delta x_j)) \\
            &\leq U(g) + \frac{\varepsilon}{2} + \Delta x_j 2M \\
            &< U(g) + \frac{\varepsilon}{2} + \frac{\varepsilon}{2} \\
            &=  U(g) + \varepsilon
        \end{align*}
        where $M$ is an upper boundon $f$ and $g$.
        So we have found a partition such that $U(f,P) <  U(g) + \varepsilon$.
        Hence, $U(f) = \inf\{ U(f,P) : P \in \mathcal{P} \} = U(g)$.

        Since $f$ and $g$ are integrable then we easily get 
        \begin{align*}
            L(f)=U(f)=U(g)=L(g)
        \end{align*}
        and we can conclude that the integral is unchanged.
    \end{proof}

    \item
    \begin{proof}
        We can proceed by induction:
        If we order the set of points which will change from $f$ to $g$
        and we denote f after the first $n$ changes by $g_n$ then
        for the base case consider if a single point is changed then 
        $\int g_1 = \int f$ by (a).

        For $n>1$ $\int g_n = \int g_{n-1} = \dots \int g_1 = \int f$ 
        where we have used a strong induction on $n$. Hence, the result 
        follows.
    \end{proof}

    \item
    The rationals are countable so we can construct dirichlet's function
    by starting at the zero function we can change each rational point to one.
    However, the function we started with is integrable while dirichlet's
    function is not integrable.
\end{enumerate}

\ex{5}
\begin{proof}
    The set $\{1/n\}$ are isolated points of $[0,1]$ so for any interval 
    the infinum of $f$ on the set will be zero. Hence, for any parition 
    $L(f,P) = 0 = L(f)$ so we focus on $U(f,P)$.

    We will need to carefully construct a sequence of partitions $P_n$
    where $U(f, P_n) - L(f, P_n) \rightarrow 0$. 

    For discontinuity $p_i = 1/i$ we can choose neighbourhood 
    \begin{align*}
        \delta_i = \frac{1}{i} - \frac{1}{i+1} = \frac{1}{i(i+1)} 
    \end{align*}
    which will only contain $p_i$.

    Next we create sequence of paritions where 
    \begin{align*}
        P_n = \{ 1 > 1-\frac{\delta_1}{n} > \frac{1}{2}+\frac{\delta_2}{n} > \frac{1}{2}-\frac{\delta_2}{n} > \frac{1}{3}+\frac{\delta_3}{n} > \dots\}
    \end{align*}
    Each $P_n$ will only parition the first $n$ discontinuities, for example
    \begin{gather*}
        P_1 = \{ 1 > 1-\frac{\delta_1}{n} > 0 \} \\
        P_2 = \{ 1 > 1-\frac{\delta_1}{n}  > \frac{1}{2}+\frac{\delta_2}{n} > \frac{1}{2}-\frac{\delta_2}{n} > 0\} \\
        \vdots
    \end{gather*}
    so $P_{n+1}$ is a refinement of $P_{n}$. 
    Hence, with some thought each parition is given by  
    \begin{align*}
        U(f,P_n) &= \frac{\delta_1}{n} + \sum_{i=2}^N \frac{2\delta_i}{n} + (\frac{1}{n} - \frac{\delta_n}{n}) \\
        &= \frac{1}{2n} + \frac{2}{n} \sum_{i=2}^n \delta_i + (\frac{1}{n} - \frac{1}{n^2 (n+1)}) \\
        &= \frac{1}{n+1} + \frac{3n-2}{2n^2}
    \end{align*}
    So 
    \begin{align*}
        U(f,P_n) - L(f,P_n) = \frac{1}{n+1} + \frac{3n-2}{2n^2}
    \end{align*}
    Taking the limit we clearly see 
    \begin{align*}
        \lim_n [U(f,P_n) - L(f,P_n)] = 0
    \end{align*}
    Hence, $f$ is integrable. This implies $U(f)=L(f)=0$ so $\int_0^1 f = 0$.
\end{proof}

\ex{6}
\begin{enumerate}[label=(\alph*)]
    \item 
    \begin{proof}
        Choose finite open cover of the set of discontinuities s.t. 
        \begin{align*}
            \sum |O_n| \leq \frac{\varepsilon}{4M}
        \end{align*}
        where $M$ is an upper bound on $f$.

        Now $f$ will be integrable on $[a,b] \backslash \{O_n\}$ since 
        $f$ is continuous. So for each closed subinterval we choose a 
        parition $P_i$. Let $P=\cup_i P_i$ then observe,
         for any partition
        \begin{align*}
            U(f,P) - L(f,P) &\leq \sum U(f,P_i) - L(f,P_i) + \sum_{n=1}^N 2M |O_n| \\
            &\leq \sum U(f,P_i) - L(f,P_i) + \frac{\varepsilon}{2}
        \end{align*}
        Now we can choose each partition such that 
        \begin{align*}
            \sum U(f,P_i) - L(f,P_i)  < \varepsilon/2
        \end{align*}
        whcih implies $ U(f,P) - L(f,P)<\varepsilon/2$. Since $\varepsilon$
        was arbitrarily then $f$ is integrable.
    \end{proof}

    \item
    \begin{proof}
        Denote each point in the finite set under consideration by $p_n$ and 
        let $k$ be the cardinalily of the set then 
        let
        \begin{align*}
            O_n =(p_n - \delta, p_n + \delta)
        \end{align*}
        where 
        \begin{align*}
            \delta \leq \frac{\varepsilon}{\sum_{n=1}^k 2}
        \end{align*}
        Hence,
        \begin{align*}
            \sum_n |O_n| = \sum_{n=1}^k 2\delta \leq \varepsilon
        \end{align*}
    \end{proof}

    \item 
    \begin{proof}
        Now $C\subset C_n$ so let's find an finite open cover of 
        $C_n$. Construct the open cover as follows: to cover each interval 
        \begin{gather*}
            O_1 = (0,\frac{1}{3^n}) \\
            O_2 = (\frac{1}{3^n},\frac{2}{3^n}) \\
            \vdots
        \end{gather*}
        there will be $2^n$ closed intervals in $C_n$. It is easy to see 
        each open set has length 
        \begin{align*}
            |O_i| = \frac{1}{3^n}
        \end{align*}
        We can cover the end points of each subinterval of $C_n$ we include 
        open sets of length $1/3^n$ centered around each end point, of which 
        there are $2^n$ so we have in total $2\times 2^n$ open intervals whcih 
        completely cover $C_n$ i.e. 
        \begin{align*}
            C_n \subset \cup_{i=1}^{2 \times 2^n} O_i
        \end{align*}
        It is clear 
        \begin{align*}
            \sum |O_i| = \frac{1}{3^n} \times 2\times 2^n = 2(\frac{2}{3})^n
        \end{align*}
        Hence, if we choose 
        \begin{align*}
            n\geq \frac{\log (\varepsilon/2)}{\log(2/3)}
        \end{align*}
        then 
        \begin{align*}
            \sum |O_i| \leq \varepsilon
        \end{align*}
        so the cantor set indeed has content zero.
    \end{proof}

    \item 
    \begin{proof}
        $h$ has discontinuities of content zero so $h$ is integrable. 
        We know for any interval the infimum of $h$ on the interval will 
        be zero so $L(h)=0$. However, since $h$ is integrable then this 
        must be the value of the integral so $\int_0^1 h = 0$.
    \end{proof}
\end{enumerate}


